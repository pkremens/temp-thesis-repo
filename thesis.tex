\documentclass[122pt,oneside]{fithesis}
% \documentclass[11pt,draft,oneside]{fithesis} puvodni nastaveni, draft nedovilil obrazky
\usepackage[plainpages=false, pdfpagelabels, draft=false]{hyperref}
\usepackage{graphicx} % obrazky
\usepackage{listings} % zdrojovy kod
\usepackage{hyperref} % url ref
\usepackage{float} % H u obrazku                             
\usepackage{enumitem} % mezery mezi Item
\usepackage[czech]{babel}                                      
\usepackage[T1]{fontenc}
%\usepackage[cp1250]{inputenc}
\textwidth 6.2in % see http://faculty.uoit.ca/bohun/latex/margins.html
\oddsidemargin 0.0in
\setcounter{tocdepth}{5} % subsubsection bude v obsahu
\setcounter{secnumdepth}{5} % subsection bude met cislo
% \addtolength{\textwidth}{2cm}
% \addtolength{\hoffset}{-1cm}
\addtolength{\voffset}{-1cm}
\addtolength{\textheight}{2cm}
\renewcommand{\figurename}{Obrázek} % zmena popisu pod obrazkem
\renewcommand{\chaptername}{Kapitola}
\renewcommand{\bibname}{Literatura}
\renewcommand{\appendixname}{Příloha}
\renewcommand{\contentsname}{Obsah}
\sloppy % slova nepresahuji okraje
\lstset{ % nastaveni formatu zdrojaku
basicstyle=\footnotesize\tt,
identifierstyle=,
showstringspaces=false,
tabsize=3,
keywordstyle=,
language=java
}
\thesistitle{Java EE aplikace pro výkonnostní testování}
\thesissubtitle{Diplomová práce}
\thesisstudent{Bc. Petr Křemenský}

\thesiswoman{false}
\thesisfaculty{fi}
\thesisyear{podzim 2012}
\thesisadvisor{Mgr. Jiří Kolář}

\begin{document}
\FrontMatter
\ThesisTitlePage

\begin{ThesisDeclaration}
\DeclarationText
\AdvisorName
\end{ThesisDeclaration}

\begin{ThesisThanks}
Na tomto místě bych rád poděkoval panu Ing. Pavlu Slavíčkovi za cenné rady a připomínky, panu Mgr. Jiřímu Kolářovi za odborné vedení práce a kolegovi Bc. Janu Martiškovi za poskytnutí inspirace při tvorbě skriptů na měření paměti serverů.

Dále děkuji své sestře Mgr. Lence Liškové a své matce Lence Křemenské za neocenitelnou pomoc při jazykové korektuře. 
\end{ThesisThanks}

\begin{ThesisAbstract}
Cílem práce je prostudovat problematiku výkonnostního testování, prozkoumat stávající řešení, identifikovat a popsat problémy s tím související  a na základě získaných znalostí navrhnout a implementovat komplexní benchmarkovou aplikaci pro měření a výkon aplikačního serveru. Aplikace bude schopná běžet dlouhodobě a testovat jednotlivé komponenty izolovaně. Aplikace bude dále v práci použita na provedení kompletní sady výkonnostních testů.
\end{ThesisAbstract}

\begin{ThesisAbstracten}
The main goal of the thesis is to study issues of performance testing, to investigate the present solutions, to identify and describe related problems and to use obtained knowledge to design and implement a complex benchmark application for performance testing of application servers. Main features of the application will be able to run for a long period of time and able to test each component in isolation. Application will be used to perform series of performance tests.
\end{ThesisAbstracten}

\begin{ThesisKeyWords}
výkonnostní testování, aplikační server, benchmarking, generování zátěže, Java EE 6, JPA, EJB, JSF, JMS, 
\end{ThesisKeyWords}

\MainMatter
\tableofcontents

\chapter*{Úvod}
Úspěch velkých firem moderní doby žije a skomírá s výkonností aplikací, které jsou pro jejich podnikání kritické. Bohužel, mnoho aplikací je používáno, aniž je adekvátně testována jejich škálovatelnost a výkonnost. Proces výkonnostního testování pomáhá identifikovat slabiny aplikace ještě předtím, než je nasazena do produkčního prostředí. Během výkonnostního testování simulujeme práci klientů na testované aplikaci a monitorujeme její chování. Cílem výkonnostního testování není hledat chyby, ale eliminovat místa, která negativně ovlivňují výkonnost aplikace a zároveň určit počáteční hodnoty klíčových cílů výkonnosti, které budou použity jako základ budoucího regresního testování.

Aplikační servery jsou páteřní komponentou pro chod mnoha společností. Umožňují rychlé a spolehlivé sdílení dat, centralizovanou distribuci software, jednoduchou modernizaci, vysokou míru zabezpečení a zvyšují efektivitu všech firemních procesů. Abychom se ujistili, že aplikační server plní správně svoji funkci, musí být i on podroben testům. Jednotlivé funkce aplikačního serveru jsou testovány pomocí jednotkových testů, avšak k měření výkonnosti musíme být schopni server nejprve řádně za(vy)tížit. K tomu nám slouží benchmarkové aplikace, díky kterým můžeme vyvinout zátěž na požadované komponenty a zkoumat jejich chování v podmínkách, které se co nejvíce blíží těm produkčním.

V úvodní kapitole definujeme metriky měření výkonu a popíšeme si klíčové cíle výkonnosti společně s procesem, při kterém nabývají konkrétních hodnot. Ukážeme si, jakým způsobem lze identifikovat kritické scénáře aplikace, vysvětlíme si rozdíly mezi jednotlivými typy testů a řekneme si něco málo o nástrojích, které nám pomáhají s automatizací testů.

V následující kapitole si vysvětlíme pojem benchmarking, popíšeme si základní typy serverů a ukážeme si rozdíly, které se mezi nimi vyskytují. Představíme si Java EE plný profil pro aplikační servery, seznámíme se s existujícími možnostmi v oblasti benchmarkových aplikace a závěr kapitoly věnujeme možnostem optimalizace aplikačního serveru.

Třetí kapitola je věnována popisu architektury aplikace a seznámení se se základními principy Java EE 6 platformy. Dále jsou podrobněji představeny některé specifikace, které jsou v aplikaci použity.

V následujících čtyřech kapitolách projdeme všemi vrstvami aplikace od persistentní až po prezentační a seznámíme se s některými podpůrnými technologiemi. V každé kapitole budou popsány technologie, které byly v aplikaci použity společně s ukázkami kódu a některými řešeními.

Závěrečná kapitola je věnována několika testům a srovnání aplikačních serverů. Nejprve jsou představeny nástroje, které budou k testům použity, a poté jednotlivé testy.

Práce obsahuje dvě přílohy, z nichž první popisuje způsob, jakým je možné aplikaci nasadit na aplikační server a druhá nabízí seznam funkcí dostupných z webového rozhraní aplikace.

\chapter{Výkonnostní testování}

Kdy se dá naše aplikace považovat za skutečně výkonnou? Z pohledu uživatele je odpověď na tuto otázku překvapivě snadná. Dobře odladěná aplikace je taková, která nám dovolí vykonat požadovanou posloupnost akcí bez pocitu frustrace způsobeného pomalou odezvou aplikace nebo jejím nepředvídaným chováním. Avšak k tomu, abychom mohli měřit výkonnost aplikace, musíme být schopni definovat, čeho chceme dosáhnout~\cite{molyneaux09}.

\section{Metriky měření výkonu}
Předtím, než budou rozebrány jednotlivé požadavky a nároky, které můžeme klást na výkonnost naší aplikace, definujme klíčové metriky, které se k měření výkonnosti aplikace běžně používají. Tyto metriky lze rozdělit na metriky orientované na službu a metriky orientované na výkonnost.
\subsection{Metriky orientované na službu}
U metrik orientovaných na službu nás zajímá, jak se aplikace chová jako služba z pohledu klienta. Patří zde: 

\vspace{5 mm}\noindent{\bf Čas odezvy}
\\\indent Představuje čas, který trvá aplikaci odpovědět na požadavek klienta. Tuto metriku nelze obecně považovat za nejdůležitější, ale na straně klienta jsou jakékoliv nedostatky v této oblasti na naší aplikaci nejevidentnější. Pomalá odezva může působit neblaze na konkurenceschopnost aplikace. Klienti považují rychlou odezvu aplikace za samozřejmost, proto v nich sebemenší opoždění zanechá silně negativní dojem. Platí, že hodnota vnímaná člověkem jako průměrná odezva, není skutečným průměrem, ale hodnotou, která je větší než 90 \% všech jednotlivých odezev~\cite{haines06}.

\vspace{5 mm}\noindent{\bf Dostupnost}
\\\indent Představuje množství času, po který je aplikace klientovi k dispozici. Je důležité, aby zbytek aplikace dokázal reagovat po selhání některé z komponent. I nepatrné množství času, po který bude aplikace nedostupná, může pro velké firmy znamenat značné ztráty.

\subsection{Metriky orientované na výkonnost}
U metrik orientovaných na výkonnost nás zajímá, jakým způsobem aplikace využívá zdroje prostředí, ve kterém běží

\vspace{5 mm}\noindent{\bf Propustonst}
\\\indent Představuje počet akcí, které lze dokončit v určitém časovém okamžiku. Vysoká míra propustnosti odráží naši schopnost rychle a efektivně obsluhovat požadavky klientů, což se pozitivně projeví na celkové výkonnosti naší aplikace.

\vspace{5 mm}\noindent{\bf Efektivita práce se zdroji}
\\\indent Představuje poměr využití systémových zdrojů s jejich celkovou kapacitou. Dobře navržená aplikace by měla se svými zdroji nakládat co nejefektivněji, aby byla schopna obsloužit co největší počet požadavků v co nejkratším čase, a přitom svými zdroji zbytečně neplýtvala.

\section{Klíčové cíle výkonnosti}
Výkonnost počítačových aplikací je velmi komplexní téma. Obecně platí, že celková výkonnost aplikace je rovna součtu výkonnosti jednotlivých částí. Je velmi důležité pochopit, že výkonnost naší aplikace se dá odladit jen do takové míry, do jaké se dají odladit jednotlivé části prostředí, ve kterém naše aplikace poběží. 

Projděme si například strukturu Java EE aplikace~\footnote{Java Enterprise Edition je součást platformy Java, určená pro vývoj a provoz podnikových aplikací a informačních systémů. Základem pro platformu Java EE je platforma Java SE, nad ní jsou definovány součásti tvořící Javu EE~\cite{wikiJavaEE}.}, která je nasazená na aplikačním serveru. Každá aplikace napsaná v jazyce Java potřebuje ke svému běhu Java virtuální stroj (JVM), jehož použití zaručuje jazyku platformovou nezávislost. Tento virtuální stroj poběží na počítači s určitým operačním systémem. Počítač komunikuje s ostatními počítači prostřednictvím sítě složené z programových a fyzických částí. Každá z těchto složek má mnoho konfigurovatelných vlastností a poskytuje mnoho možností, jak ji optimalizovat. Každá z nich se zároveň chová jinak v různých podmínkách. Z uvedených skutečností vyplývá, že k tomu, aby výsledky našich testů měly nějakou reálnou hodnotu, je potřeba naši aplikaci testovat v podmínkách, které se budou co nejvíce podobat těm, ve kterých bude skutečně nasazena.

Jelikož je každá aplikace unikátní, jsou unikátní také požadavky na její výkonnost. Jsou situace, kdy lze za dobrou výkonnost považovat schopnost obsloužit vysoký počet klientů, kteří k aplikaci přistupují souběžně, ale také situace (počet klientů současně využívajících aplikaci není vysoký), kdy jediné, co nás zajímá, je rychlost, s jakou dokáže naše aplikace požadavky jednotlivých klientů obsloužit.
 
Se znalostí základních metrik výkonnosti nyní můžeme definovat klíčové cíle výkonnosti (Key performance targets).

\vspace{5 mm}\noindent{\bf Dostupnost aplikace}
\\\indent Požadavek na dostupnost z pohledu klienta je celkem jednoduchý: aplikace by měla být dostupná po celou dobu s výjimkou plánovaných údržeb. To, že aplikace reaguje na požadavky klienta, nemusí nutně znamenat, že je dostupná. Aplikace je považována za nedostupnou, jakmile ji není klient schopen efektivně využívat (aplikace vrací chybové hlášky a jiné).
Z pohledu výkonnostních testů nás zajímá dostupnost aplikace s rostoucí zátěží (počet klientů souběžně přistupujících k aplikaci, objem dat nutných ke zpracování). 

\vspace{5 mm}\noindent{\bf Čas odezvy}
\\\indent Čas odezvy se dá měřit více způsoby. Aby nedocházelo k nejednoznačnostem, mělo by být jasně definováno, jak měřit čas odezvy ještě před formulováním požadavků. U moderních webových aplikací se čas odezvy počítá jako součet času odezvy serveru, času, který data putují sítí, a času, který zabere klientovi zpracovat a zobrazit požadovaný obsah. Nicméně z pohledu výkonnostního testování je za čas odezvy považována častěji hodnota získaná vypuštěním doby na zpracování a zobrazení klientem. Toto je způsobeno tím, že nástroje na generování zátěže, které jsou při výkonnostních testech hojně používány, často tyto hodnoty ignorují~\cite{moorthy09}.
V závislosti na povaze testu nás mohou zajímat:
\begin{itemize}
  \item čas odezvy požadavku na zobrazení stránky
  \item čas odezvy zpracování jednoduché transakce
  \item čas odezvy komplexních scénářů
  \item průměrný čas odezvy požadavku
  \item poměr splnění a nesplnění stanovené hodnoty času odezvy
  \item agregovaný čas odezvy (suma průměrných časů odezev všech požadavků v daném scénáři podělena počtem scénářů)
\end{itemize}


\vspace{5 mm}\noindent{\bf Propustnost aplikace}
\\\indent Z pohledu Java EE aplikace nás většinou zajímá počet požadavků (transakcí), které jsme schopni obsloužit za daný časový úsek. V typické aplikaci bude více klientů vykonávat různorodé akce souběžně, proto nás zajímá propustnost jednotlivých požadavků i různých skupin složitějších scénářů (simulace reálného vytížení aplikace). Cílem testů je monitorovat chování aplikace při zvyšujícím se počtu souběžných požadavků a nalézt bod saturace, tj. počet požadavků, při nichž dojde k zahlcení aplikace, což se projeví například náhlým zhoršením času odezvy aplikace. 

\vspace{5 mm}\noindent{\bf Využití zdrojů}
\\\indent Výkonnost aplikace může být negativně ovlivněna šířkou pásma mezi klientem a aplikací, proto je potřeba určit práh objemu dat posílaných naší aplikací klienty a ten nepřekračovat. Z hlediska testování využití šířky pásma nás zajímá následující:
\begin{itemize}
  \item objem dat procházejících sítí
  \item datová propustnost - jedním z naších cílů může být dosáhnutí určitého datového toku, který by měla být naše aplikace schopna zpracovat
  \item chybovost dat - vysoký počet chyb při přenosu vyžaduje přeposílání, čímž samozřejmě dochází k degradaci výkonnosti aplikace
\end{itemize}
V podobném smyslu bychom měli také definovat limity zdrojů serveru, které by naše aplikace neměla překračovat. Zde se zajímáme převážně o:
\begin{itemize}
  \item vytížení CPU
  \item využití paměti
  \item frekvenci V/V operací
  \item a jiných
\end{itemize}

\section{SLA}
V ideálním světě bychom měli kontrolu nad každou částí prostředí, ve kterém naše aplikace poběží, bohužel v reálných podmínkách se často musíme spolehnout na externí dodavatele, proto je nutné klást si reálné cíle výkonnosti. Pro definování těchto cílů se používají SLA (Service Level Agreement). Předtím, než jsou SLA přesně definovány, je jakékoliv testování jen ztrátou času, jelikož v důsledku unikátnosti každé aplikace neexistují žádné pevně dané hodnoty, se kterými bychom mohli naši aplikaci srovnávat, a tudíž dělat o její výkonnosti jakékoliv závěry.

Na vytváření SLA se podílejí hlavně technický (ATO) a obchodní (ABO) ředitel aplikace. ABO provede studii podnikových plánů zákazníka a navrhne SLA, která by měla být z pohledu zákazníka splněna, tímto se zároveň zaručí, že pokud budou daná SLA splněna, budou splněny také požadavky zákazníka. ATO analyzuje SLA a určí potřebné technické požadavky, které jsou nutné k jejich splnění. Je velice důležité, aby se na definováni SLA podílely obě strany. Jestliže by probíhala definice SLA pouze ze strany ABO, mohlo by se stát, že budou definovány nedosažitelné požadavky, naopak pokud by se na definici podílela pouze strana ATO, aplikace by mohla být po nasazení do reálných podmínek pro koncové klienty nevyhovující~\cite{haines06}.

Efektivní SLA musí být specifické, flexibilní a realistické.

\vspace{5 mm}\noindent{\bf Specifické SLA}
\\\indent Cíle určené v SLA by měly mít přesné hodnoty. Je nemožné později rozhodnout, jestli naměřená hodnota 2,25 sekundy splňuje zadání požadující hodnotu “okolo 2 sekund”. ABO i ATO by se měli dohodnout na konkrétních hodnotách, což nám zaručí spokojenost na straně klientů a testující tým bude znát přesné hodnoty k testování ještě před nasazením aplikace.

\vspace{5 mm}\noindent{\bf Flexibilní SLA}
\\\indent Flexibility SLA lze dosáhnout určením míry požadovaného úspěchu a tím brát v potaz nesplnění podmínek v důsledku nepředvídatelných okolností. Bude-li naše aplikace v jednom případě ze dvaceti požadavků odpovídat o něco pomaleji, dá se to většinou klientů akceptovat. Ani s flexibilitou se to však nesmí přehánět a to, o kolik pomaleji může naše aplikace v oněch pár případech odpovídat, musí být přesně definováno.

\vspace{5 mm}\noindent{\bf Realistické SLA}
\\\indent Dosáhnutí reálných cílů v SLA se dá zajistit právě tím, že se na něm shodnou ABO i ATO. Často dochází k situacím, kdy jsou SLA definovány pouze ABO bez souhlasu ATO. Poté, co se dostanou požadavky k ATO, jsou jednoduše ignorovány pro jejich nesplnitelnost.


\section{Identifikace a příprava kritických scénářů}
V této části se budeme věnovat hledání co nejvěrnější simulace aplikační zátěže, tj. spouštění běžných akcí klientů ve správném poměru (např. přihlášení se do systému, vyhledání určité položky v katalogu a odhlášení se). Je dobré pouštět testy s různým typem zátěže, abychom zjistili, jak se mění charakteristika výkonu aplikace s měnící se podobou zátěže. K vytvoření zátěže lze využít některý z automatizovaných nástrojů. Dříve, než začneme psát testy, je potřeba si rozmyslet, z jakých akcí se bude zátěž skládat.

\subsection{Požadavky na aplikaci}

Předtím, než se pustíme do vytváření skriptů (scénářů) pro naše testy, musíme si být jisti, že je naše aplikace dostatečně stabilní pro výkonnostní testování. Je na místě zopakovat, že cílem výkonnostního testování není hledání chyb v aplikaci, proto by se nemělo stát, že čas vymezený na testování výkonu je spotřebován při opravě chyb. Aplikace je považována za stabilní, pracuje-li podle očekávání. Jsou-li známy nějaké významnější funkční nedostatky ještě před testováním (např. zboží se přidá do košíku jen v 8 z 10 případů), je lepší věnovat čas nejprve odstranění těchto nedostatků než testování, protože by se lehce mohlo stát, že tyto nedostatky budou maskovat jiné chyby, které by se během testů teprve objevily.

Máme-li k dispozici dostatečně stabilní aplikaci, je potřeba si zajistit tzv. zmražení kódu. Jakákoliv změna kódu může zásadně ovlivňovat výsledky testů, dokonce může mít fatální důsledek na jejich funkčnost (skripty jsou závislé na konkrétní verzi kódu, protože na specifický požadavek očekávají specifickou odpověď). Je-li během výkonnostních testů nalezen kus kódu, který vyžaduje změnu, je potřeba zajistit, aby vývojáři tuto změnu provedli až po ukončení testovacího cyklu. Jestliže je z nějakého důvodu potřeba tuto změnu provést ihned, je zásadní, aby se o tom dozvěděl testující tým~\cite{molyneaux09}.

\subsection{Určení cílové zátěže}

Nyní již předpokládejme, že je naše aplikace připravena k testování. Existuje několik způsobů, jakými lze postupovat při vybírání vhodné zátěže~\cite{barber11}.

Nejedná-li o první generaci aplikace, je průzkum historie používání předchozí verze aplikace nejsnadnější a nejefektivnější cestou, jak získat věrohodná data o zátěži, se kterou je nutné počítat.

Jestliže taková data nemáme, je možné je získat od konkurence, jedná-li se o podobný typ aplikace (společnosti se rády chlubí dosaženými výsledky, jsou-li příznivé).

Nemáme-li k dispozici ani jedno, tak nám nezbývá než zátěž odhadnout na základě marketingových předpokladů.

Při provádění průzkumu dat, ať už historických nebo dat konkurence, by nás mělo zajímat hlavně:
\begin{itemize}
  \item maximální počet uživatelských sezení, která běžela souběžně
  \item maximální počet prohlížených stránek souběžně
  \item průměrná délka trvání jednoho sezení
  \item průměrný počet stránek, které uživatel během jednoho sezení shlédne
\end{itemize}

\subsection{Distribuce zátěže mezi jednotlivé funkce}

Nyní, když jsme určili celkovou zátěž, kterou budeme testovat, je potřeba ji vhodně distribuovat mezi jednotlivé transakce, které naše aplikace nabízí. Termínem transakce se v tomto případě myslí jakákoliv operace, kterou může uživatel s naší aplikací vykonat. I v tomto případě je nejlepší cestou využít historii předchozí verze aplikace, je-li k dispozici, v opačném případě nezbývá než rozložení zátěže mezi transakce opět odhadnout. 

\begin{figure}[!ht]
\centering
\includegraphics[width=1.0\textwidth]{tabulkanavigace}
\caption{Vybrané scénáře prezentovány pomocí tabulky~\cite{molyneaux09}}
\label{img:tabulkaNavigace}
\end{figure}

Jedním z možných řešení je transakce rozdělit do následujících (ne nutně disjunktních) skupin:
\begin{itemize}
  \item {\bf běžné}
  \item {\bf náročné} - zejména na zdroje
  \item {\bf kritické} - zásadní pro provoz
  \item {\bf právně vymahatelné} - SLA
  \item {\bf zjevné} - uživatel si okamžitě všimne, je-li něco špatně
  \item {\bf technicky riskantní} - místa, která ráda selhávají
\end{itemize}

Do každé skupiny vybereme nejvýše 5 transakcí (raději méně než více, je nutné opět zopakovat, že se nejedná o testy funkcionality, není proto nutné testovat úplně všechny transakce, které jsou k dispozici), čímž bychom měli získat odpovídající množinu kritických scénářů pro naše testy. Výsledné scénáře lze prezentovat buď pomocí tabulky (viz obrázek~\ref{img:tabulkaNavigace}), nebo pomocí diagramu (viz obrázek~\ref{img:diagramNavigace}).

\begin{figure}[!ht]
\centering
\includegraphics[width=0.8\textwidth]{diagramnavigace}
\caption{Vybrané scénáře prezentovány pomocí diagramu}
\label{img:diagramNavigace}
\end{figure}

\section{Typy testů}
Existuje několik typů testů, které se liší svým průběhem i účelem za kterých je provádíme. Představme si proto nejdůležitější z nich~\cite{loadTest}.

\vspace{5 mm}\noindent{\bf Základní test}
\\\indent Slouží k ustanovení počátečních hodnot, které mohou být později použity pro porovnávání s výsledky ostatních testů. Test je běžně prováděn za použití jediného virtuálního uživatele pro předem určený počet akcí. Díky těmto hodnotám můžeme sledovat míru degradace výkonnosti aplikace v závislosti na měnících se metrikách testu (např. počet virtuálních uživatelů).

\vspace{5 mm}\noindent{\bf Zátěžový test}
\\\indent Slouží k určení výkonnosti aplikace při běhu s požadovanou zátěží, kterou získáváme jako odhad skutečného vytížení aplikace v reálných podmínkách. Test probíhá přibližně 5 minut, během nichž by měla být křivka zátěže na aplikaci konstantní. Cílem testu je rozhodnout, zda-li naše aplikace splňuje hodnoty určené v SLA, a určit minimální požadavky na běhové prostředí.

\vspace{5 mm}\noindent{\bf Test hraniční zátěže}
\\\indent Zátěž na aplikaci je postupně zvyšována, díky čemuž můžeme sledovat chování aplikace v momentech, kdy počet uživatelů neustále roste a systémové zdroje začínají docházet. Cílem tohoto testu je nalezení horní hranice zátěže (čas odezvy převýší akceptovatelné maximum, pád aplikace nebo části podpůrné infrastruktury, a jiných), kterou je aplikace schopna unést. Po dosažení horní hranice snižujeme zátěž, abychom zjistili, zda-li je aplikace schopna se opět vzpamatovat. Na rozdíl od zátěžových testů se nesnažíme o simulaci reálné zátěže, ale aplikaci záměrně přetěžujeme.

\vspace{5 mm}\noindent{\bf Test odolnosti}
\\\indent Test odolnosti je specifický svou délkou. Ta se běžně pohybuje v rozmezí dnů (testy odolnosti aplikací, na jejichž funkcionalitě závisí lidské životy mohou trvat i několik měsíců). Cílem je ujistit se, že se neobjeví nové problémy (chyby, paměťové úniky, limit počtu transakcí, a jiné) po delší době běhu aplikace, které se neprojevily u předchozích testů kvůli jejich relativně krátkým dobám běhu. Tyto defekty se projeví buď postupným zvýšením doby odezvy, nebo náhlou ztrátou dostupnosti aplikace (po dosažení určitého počtu transakcí).

\vspace{5 mm}\noindent{\bf Test selhání}
\\\indent Během testu selhání zkoumáme chování redundantních mechanismů při selhání jednoho z nich. Příkladem může být test webového prostředí, které používá více serverů a jeden z nich náhle selže. Cílem testu je zjistit, jak se s tímto selháním vypořádají ostatní komponenty.
 
\vspace{5 mm}\noindent{\bf Test výkonnosti}
\\\indent Test výkonnosti probíhá za relativně malé zátěže. Měl by předcházet ostatním testům. Pomůže nám včas identifikovat vážné nedostatky na výkonnosti naší aplikace, ale také zásadní nedostatky na běhovém prostředí. Tyto testy jsou používány také pro porovnávání výkonu mezi jednotlivými verzemi produktu.

\section{Způsoby budování zátěže}

Kromě počtu klientů, kteří budou k naší aplikací přistupovat souběžně, můžeme běh testu ovlivnit také způsobem, jakým budeme klienty do aplikace přidávat~\cite{molyneaux09}.

\vspace{5 mm}\noindent{\bf Velký třesk}
\\\indent Všichni klienti přistupují k aplikaci ve stejný čas a pracují souběžně, ale nikoliv synchronně (vykonávají různé scénáře).

\vspace{5 mm}\noindent{\bf Postupné navyšování}
\\\indent Začínáme s jedním klientem a další je přidán vždy po uplynutí definovaného časového intervalu. Tento postup je opakován, dokud není dosaženo požadované zátěže, je ovšem nutné zvolit vhodný interval mezi přidáním jednotlivých klientů, aby nebylo věnováno více času budování zátěže než samotnému testování.

\vspace{5 mm}\noindent{\bf Postupné navyšování s mezikroky}
\\\indent Tato varianta rozšiřuje tu předchozí o schopnost zachytit chování aplikace v určitých etapách budování zátěže. Příkladem by mohlo být postupné navyšování na 1000 klientů s tím, že v momentě, kdy jich bude k aplikaci souběžně přistupovat 300 (resp. 700), bude generování na nějaký čas pozastaveno, což nám umožní sledovat chování aplikace i v těchto mezikrocích.

\vspace{5 mm}\noindent{\bf Postupné navyšování (s mezikroky), postupné snižování (s mezikroky)}
\\\indent Po dosažení požadované hodnoty je zátěž na systém postupně snižována. Různé hodnoty naměřené v mezních hodnotách mohou indikovat paměťové úniky nebo jiné defekty na aplikaci.

\section{Nástroje pro budování zátěže}

Pro usnadnění testování je nutné vybrat vhodný testovací nástroj. V současnosti je k dispozici řada komerčních variant (IBM Rational Performance Tester, Blitz), ale také nespočet řešení s otevřenými zdrojovými kódy (Apache JMeter, Faban, Hammerora). Tyto nástroje nám usnadňují všechny etapy testovacího cyklu od návrhu testů až po zpracování naměřených dat. 

Testovací nástroje běžně poskytují následující funkcionalitu:

\vspace{5 mm}\noindent{\bf Skriptovací modul}
\\\indent Umožňuje zaznamenání aktivity klienta. Každou operaci, kterou klient vykoná, je možné později modifikovat (změnit hodnoty zadané klientem, upravit časování, a jiné).

\vspace{5 mm}\noindent{\bf Modul na správu testů}
\\\indent Umožňuje vytvářet složitější scénáře, které jsou složeny z různých aktivit klientů. Pomocí generátorů zátěže jsou tyto scénáře využívány k vytváření zátěže na aplikaci.

\vspace{5 mm}\noindent{\bf Modul na generování zátěže}
\\\indent Mají-li testy plnit svůj účel, musí existovat možnost je libovolně opakovat s tím, že naměřené hodnoty by se měly při zachování stávajících podmínek lišit pouze minimálně. K~dosažení této konzistence je potřeba vyloučit lidské uživatele a nahradit je virtuálními. Tento problém pomáhají řešit nástroje na generování zátěže (viz Obrázek~\ref{img:jmeter}).

\begin{figure}[!ht]
\centering
\includegraphics[width=0.7\textwidth]{jmeter}
\caption{Modul na generování zátěže nástroje Apache JMeter}
\label{img:jmeter}
\end{figure}

\vspace{5 mm}\noindent{\bf Analytický modul}
\\\indent Poskytuje nám schopnost sbírat a vyhodnocovat hodnoty naměřené v každém testu. Vyhodnocením se rozumí například vykreslení grafů, znázornění důležitých hodnot v tabulkách, a jiné.

\section{Co nelze opomenout}
Nyní zmíníme pár věcí, které nesmíme při používání těchto nástrojů opomenout.

\vspace{5 mm}\noindent{\bf Latence}
\\\indent Přistupujeme-li k nějakému vzdálenému zdroji (webová stránka, webová služba, aj.), musíme počítat s určitou odezvou. Latencí nazýváme čas, který trvá požadavku, než doputuje k serveru a zase zpátky (nepočítáme čas, který potřebuje server na zpracování). S latencí můžeme mít problémy, využíváme-li vzdálená datová centra. S tímto je potřeba počítat už při formulování SLA a zahrnout latenci i do našich testů~\cite{bhatt09}.

\vspace{5 mm}\noindent{\bf Navigace po stránkách}
\\\indent Zejména u webových aplikací platí, že se dá dosáhnout jednoho cíle mnoha různými způsoby. Představme si uživatele vykonávajícího určitou transakci, který si uvědomí, že se spletl a pomocí tlačítek navigace se přesune z půlky jedné transakce na začátek jiné. Nasimulovat všechny kombinace je nemožné, protože jejich počet je nekonečný, proto bychom se o to neměli ani pokoušet. Není na škodu přidat pár scénářů, které tuto situaci simulují, ale je naprostou ztrátou času se těmto situacím intensivněji věnovat.

\vspace{5 mm}\noindent{\bf Čas na rozmyšlenou}
\\\indent Mají-li být podmínky při testování aplikace co nejvěrohodnější, je vhodné v testovacích scénářích simulovat čas, který uživatel strávil mezi jednotlivými obrazovkami. Jsou-li na tvorbu testovaných scénářů použity automatizované nástroje, jsou nám hodnoty mezi jednotlivými požadavky známy. Tyto hodnoty je potřeba brát pouze jako ukázkové, protože koncovému uživateli bude vše trvat mnohem déle, proto by měly být ještě před použitím upraveny.

Čas na rozmyšlenou se používá při provádění výkonnostních testů u hotové, funkční aplikace. Jsou-li výsledkem testů obecná měření, čas na rozmyšlenou se nepoužívá. Příkladem může být porovnání výkonnosti různých programovacích technik. Poté, co najdeme nejefektivnější implementaci dané části, můžeme ji otestovat i s časem na rozmyšlenou.

\vspace{5 mm}\noindent{\bf Počet virtuálních uživatelů se nerovná počtu aktivních uživatelů}
\\\indent Jedním z požadavků na výkonnost aplikace může být podpora určitého počtu uživatelů pracujících s aplikací souběžně. Ovšem je potřeba rozlišovat mezi počtem virtuálních uživatelů (počet uživatelů z pohledu testovacího nástroje) a počtem aktivních uživatelů (počet uživatelů využívajících aplikaci). Musí-li se uživatel před použitím aplikace přihlásit a po použití zase odhlásit (v opačném případě je považován za aktivního během celého testu), znamená to, že je během jisté doby testu odhlášen a tím pádem se nedá počítat za aktivního. Toto může být problém, máme-li za jeden z výkonnostních cílů požadavek např. 1000 uživatelů souběžně využívajících aplikaci. I přesto, že testovací nástroj bude indikovat 1000 souběžných uživatelů, může se v důsledku přihlašování a odhlašování stát, že aplikace nebude nikdy použita více než stovkami uživatelů souběžně. Což se dá vyřešit např. prodloužením času vykonávání jednotlivých scénářů, aby zůstal každý virtuální uživatel aktivní po delší dobu.

\chapter{Benchmarknig}

Pojmem benchmarking rozumíme proces porovnávání výkonnosti objektu tím, že na něm provádíme různé testy a zkoušky a naměřené hodnoty porovnáváme s výsledky měření na jiných objektech se stejnou funkcionalitou. Nové generace produktů a stále nové architektury mají za následek, že je nemožné porovnat výkonnost příbuzných produktů pouze pohledem na jejich specifikace. Chceme-li například porovnat výkonnost procesorů od dvou různých výrobců, nestačí nám srovnat jen velikosti taktovacích frekvencí, počty jader a jiné metriky, ale je nutné brát v úvahu i míru optimalizace, použitou architekturu a další faktory, které již číselně vyjádřit nelze. K tomuto nám slouží benchmarkové aplikace, které jsou speciálně navržené k tomu, aby produkovaly specifický druh zátěže na konkrétní komponentu nebo systému.

Benchmarková aplikace můžeme mít podobu reálné aplikace pro testování kompletní funkcionality systému, ale může se taky zaměřit na konkrétní komponenty (V/V operace, rychlost databáze a jiné).

Benchmarkové aplikace se používají při vývoji a testování nových komponent i při propagaci nových produktů, čehož ovšem někteří výrobci začali zneužívat. Z historie jsou známé příklady podvádění, kdy firmy vyvíjely své produkty záměrně tak, aby měly co nejlepší skóre při benchmark testech, tyto produkty ale slibovaného výkonu ve skutečných podmínkách nedosahovaly~\cite{wikiBench}.

Hledáme-li specifický produkt, nelze vybírat pouze podle výsledků benchmarkových testů. Dobré výsledky v benchmark testech nám zaručí kvalitu výpočetního výkonu, nesmíme ale opomenout:

\begin{itemize}
  \item kvalitu služeb
  \begin{itemize}
    \item zabezpečení, spolehlivost, škálovatelnost, robustnost a jiné
  \end{itemize}
  \item celkovou cenu vzhledem k dosaženému výkonu 
  \begin{itemize}
    \item  výrobci sice musí udávat vedle naměřeného skóre také poměr ceny a výkonu spočítaný podle speciálního vzorce, ale je známo, že v těchto měřeních jsou použity konfigurace speciálně navržené pouze pro benchmarking. V případě, že bychom chtěli těchto hodnot dosáhnout i my, museli bychom si připlatit.
  \end{itemize}
  \item omezení prostředí
  \begin{itemize}
    \item více výkonu znamená větší energetické nároky, což může být kritické například u mobilních zařízení
    \item server, který zabírá příliš místa, nemusí vyhovovat omezením datového centra
  \end{itemize}
\end{itemize}

\section{Servery}

Pod pojmem server rozumíme fyzický počítač, jehož výpočetní výkon je věnován běhu jedné nebo více služeb. Tyto služby jsou poskytovány ostatním strojům prostřednictvím počítačové sítě. Servery dělíme vzhledem k povaze služeb, které mohou poskytovat. Jedny z nejběžnějších typů jsou~\cite{wikiServerComp}:
\begin{itemize}
  \item {\bf aplikační server} - poskytuje běhové prostředí aplikacím
  \item {\bf databázový server} - poskytuje databázové služby ostatním počítačovým programům nebo počítačům
  \item {\bf souborový server} - poskytuje vzdálený přístup k souborům
  \item {\bf herní server} - poskytuje klientům počítačových her schopnost hrát po síti
  \item {\bf DNS server} - poskytuje službu převodu doménových jmen na IP adresy
  \item {\bf webový server} - doručuje a odesílá odpovědi na požadavek klienta prostřednictvím HTTP protokolu
\end{itemize}

V naší práci budeme dále pojem sever používat výhradně pro aplikační sever certifikovaný na Java EE plný profil. Dříve, něž si vysvětlíme princip profilů v Java EE, podíváme se podrobněji na rozdíl mezi aplikačním a webovým serverem, jelikož jsou tyto dva pojmy často zaměňovány.

\subsection{Webový server}
Hlavní funkcí webového serveru je doručovat obsah klientům, kteří o něj požádali pomocí HTTP protokolu. Obsahem může být například: statická HTML stránka, obrázek nebo kaskádový styl. Odpovědí webového serverů může být také přesměrování požadavku na jinou adresu nebo generování dynamického obsahu pomocí technologií jako JSP, servletů, JSF, ASP, PHP, JavaScriptu a jiných. Požadavek klienta na určitou stránku je proveden kontaktováním webového serveru na specifické adrese (IP adresa). Po přijetí požadavku doručí webový server obsah klientovi, kterým je nejčastěji webový prohlížeč.

Plná implementace HTTP protokolu poskytuje také funkcionalitu k příjímání obsahu od klientů. Této schopnosti je využito například při vyplňování webových formulářů nebo při nahrávání dat na server.

Co do výkonnosti nás u webových serverů zajímá zejména:
\begin{itemize}
  \item počet požadavků, které je server schopen obsloužit za určitý časový interval
  \item čas potřebný k obsloužení jednotlivých požadavků
  \item datová propustnost
\end{itemize}
Mezi populární webové servery patří například:
\begin{itemize}
  \item Apache
  \item IIS
  \item nginx
  \item GWS
\end{itemize}

\subsection{Aplikační server}
Zatímco webový server se zabývá výhradně odesíláním HTML stránek zobrazovaných na webovém prohlížeči klienta, aplikační server navíc poskytuje klientským aplikacím přístup k aplikační logice pomocí různých protokolů včetně HTTP. Data putující mezi aplikačním serverem a klientem nejsou nijak omezena a mohou mít podobu programové logiky. Ve většině případů server využívá tuto logiku skrze aplikační rozhraní komponent~\cite{sintes02}.

Aplikační server si spravuje své vlastní zdroje a nabízí služby jako například zabezpečení, zpracování transakcí, fondy zdrojů (vláken, připojení k databázi, a jiné) a JMS. Aplikační server by měl být zejména schopen obsloužit mnoho souběžně přistupujících klientů, zvládat transakce pomocí Enterprise Java Bean (EJB) v co nejkratším čase a podporovat komplexní databázové operace.


\subsubsection{Java EE plný profil}
\label{profily}
Novinkou v Java EE 6 prostředí je zavedení profilů. Jejich hlavním cílem je redukovat velikost platformy tak, aby co nejlépe vyhovovala specifickým požadavkům vývojářů. Bez profilů nezáleželo jak komplexní aplikaci jsme vyvíjeli, ale museli jsme ji nasadit na server podporující všech 28 specifikací, což bylo terčem mnohé kritiky. Profil definuje podmnožinu (v budoucnu může být i nadmnožinou, jelikož každý profil se může vyvíjet nezávisle na EE specifikaci) celé Java EE platformy. Ve verzi Java EE 6 jsou specifikovány profily dva, webový profil a Java EE 6 plný profil. K tomu, aby byl server certifikovaný na webový profil, musí podporovat následující specifikace:
\begin{itemize}
  \item Servlet 3.0
  \item JavaServer Pages (JSP) 2.2
  \item Expression Language (EL) 2.2
  \item Podpora debugování pro ostatní jazyky (JSR-45)~\footnote{Každý požadavek na změnu nebo doplnění platformy je realizován v rámci požadavku na změnu (JSR). Je-li požadavek na změnu schválen odpovídajícím výborem JCP, je vytvořena pracovní skupina, která pracuje na příslušné specifikaci. Specifikace přitom prochází různými fázemi a schvalovacími procesy definovanými v rámci JCP. Ve výborech JCP i v pracovních skupinách JSR poté pracují zástupci různých firem~\cite{wikiJSR}.} 1.0
  \item Standardní knihovna značek pro JSP (JSTL) 1.2
  \item JavaServer Faces (JSF) 2.0
  \item Běžné anotace Java platformy (JSR-250) 1.1
  \item Enterprise JavaBeans (EJB) 3.1 odlehčená verze
  \item Java Transaction API (JTA) 1.1
  \item Java Persistence API (JPA) 2.0
  \item Validace Bean 1.0
  \item Managed Beany 1.0
  \item Interceptory 1.1
  \item CDI 1.0
  \item DI 1.0
\end{itemize}

K certifikaci na Java EE 6 plný profil je potřeba, aby server podporoval kompletní Java EE platformu.
Mezi populární aplikační servery patří:
\begin{itemize}
  \item JBoss Enterprise Application Platform 6 (certifikace na webový i plný profil)
  \item Oracle GlassFish Server 3.x (certifikace na webový i plný profil)
  \item JBoss Application Server 7.x (certifikace na webový i plný profil)
  \item Apache Geronimo 3.0-beta-1 (certifikace na webový i plný profil)
  \item Caucho Resin 4.0.17 (certifikace pouze na webový profil)
  \item Apache TomEE 1.0 (certifikace na webový i plný profil)
  \item IBM WebSphere Application Server 8.x (certifikace pouze na plný profil)
\end{itemize}

\section{Benchmarking aplikačního serveru}
Kvalitní benchmarková aplikace pro aplikační servery by měla být schopna běžet na aplikačních serverech všech výrobců. Měla by simulovat reálné podmínky a vykonávat komplexní databázové transakce na co nejvíce objemných databázích. Měla by být navržena tak, aby nám umožnila testovat co nejvíce funkcí aplikačního serveru~\cite{sybase}. 

K dosažení smysluplných výsledků při benchmarkingu bychom měli postupovat dle následujících kroků:
\begin{itemize}
  \item identifikace a pochopení základní funkcionality produktu, který testujeme
  \item definice metrik na měření požadované funkcionality
  \item vytvoření testů, které budou tuto funkcionalitu vytěžovat a simulovat běh v reálných podmínkách
  \item testovat produkt ve stabilních podmínkách, tj. všechny testy by měly probíhat ve stejném prostředí
  \item měření a vyhodnocení definovaných metrik
  \item prezentace výsledků ve odpovídající formě (grafy, tabulky, a jiné)
\end{itemize}

\subsection{Benchmarkning komerčních aplikačních serverů}
K porovnání komerčních aplikačních serverů nám může  sloužit například sada benchmarkových testů vyvíjených organizací SPEC (Standard Performance Evaluation Corporation). Tato každého čtvrt roku na svých stránkách zveřejňuje výsledky testů mnoha typu produktů.

SPEC si dává za cíl poskytnout trhu férovou metriku k porovnávání různých systémů. Nejnovější verze testů pro porovnávání aplikačních serverů má v době psaní práce název SPECJEnterprise2010. SPECJEnterprise2010, umožňuje srovnávání výkonnosti Java EE 5 serverů a jejich podpůrné infrastruktury. Tato benchmarková aplikace je tvořena Java třídami, Java servlety, JSP, EJB, JPA entitami a MDB. Jedná se již o třetí generaci testů vyvíjených organizací SPEC pro porovnávání aplikačních serverů. Výsledky testů lze zdarma prohlížet na stránkách SPEC organizace.

Mezi výrobce aplikačních serverů, jichž výsledky jsou k dispozici patří:
\begin{itemize}
  \item IBM se svým WebSphere Application Server
  \item Oracle se svým Weblogic Server
\end{itemize}
Bohužel, jelikož nejsou benchmarky organizace SPEC zdarma, nejsou k dispozici výsledky serverů s otevřeným zdrojovým kódem, což nám znemožňuje jakékoliv porovnání s komerčními produkty.

\subsection{Benchmarkning aplikačních serverů s otevřeným zdrojovým kódem}
Nejistá ekonomická situace posledních let a stále se zvyšující náklady na podnikání nutí mnoho společností opustit prostředí drahých komerčních produktů a hledat náhradu v podobě aplikací s otevřeným zdrojovým kódem. Toto řešení zároveň snižuje počáteční kapitál, který je nutno investovat do začátku podnikání, což zvyšuje šance na přežití mnoha společnostem.

Základem našeho podnikání by proto měl být i kvalitní aplikační server, který poskytne našim aplikacím stabilní a výkonné běhové prostředí.

Na výběr máme více možností, mezi nejpopulárnější patří:
\begin{itemize}
  \item Apache Geronimo 3
  \item GlassFish Server 3
  \item JBoss Application Server 7
  \item Resin Application Server 4
\end{itemize}

Jedním z důvodů, proč je těžké se mezi nimi rozhodnout, je nedostupnost dat z benchmark testování, na základě kterých bychom mohli tyto servery mezi sebou porovnávat. Mezi dostupné aplikace s otevřeným zdrojovým kódem používané pro porovnávání výkonu aplikačních serverů je DayTrader momentálně spravovaný firmou Apache. 

\subsubsection{DayTrader}
DayTrader, původně vyvinut firmou IBM, simuluje online burzu s akciemi. Uživatelé se mohou zaregistrovat, nakupovat a prodávat akcie nebo sledovat stav svého portfólia. S pomocí některého z nástrojů pro generování zátěže nám DayTrader poskytuje zátěž srovnatelnou s reálnými podmínkami. DayTrader neposkytuje pouze možnost simulovat běžný provoz, umožňuje nám také testovat mnohé komponenty izolovaně (např. zobrazení jednoduché HTML stránky, zavolání servletu pro přečtení a zobrazení jednoho řádku z databáze za použití předem připraveného SQL dotazu, zaslání zprávy na dané téma za použití JMS a mnoho dalších). 

Jednou z předností DayTraderu je jeho konfigurovatelnost. DayTrader nám umožňuje vybrat si způsob, jakým bude aplikace přistupovat k databázi. K dispozici máme implementace tří různých návrhových vzorů~\cite{daytrader}.

\vspace{5 mm}\noindent{\bf Servlet-JDBC}
\\\indent Třída TradeDirect provádí CRUD operace (vytvoření, čtení, editaci a smazání) přímo za použití vlastního JDBC~\footnote{Java Database Connectivity - jednotné rozhraní pro přístup k relačním databázím} kódu. Připojení k databázi i potvrzení a rušení transakcí jsou spravovány manuálně.

\vspace{5 mm}\noindent{\bf Servlet-SessionBean-JDBC}
\\\indent TradeJDBC bezstavová session beana je použita jako obalová třída pro třídu TradeDirect.

\vspace{5 mm}\noindent{\bf Servlet-SessionBean-EntityBean}
\\\indent Bezstavová session beana TradeBean používá entity beany jako reprezentaci objektů. Stav těchto objektů je spravován EJB kontejnerem aplikačního serveru.
\newline
\\\indent Můžeme nastavit také způsob, jakým bude aplikace dokončovat nákupy a prodeje akcií. Na výběr máme synchronní a asynchronní mód. Zatímco v synchronním jsou objednávky uzavírány okamžitě, v asynchronním je každá objednávka nejprve zaznamenána a dokončena asynchronně pomocí JMS. Další z možností jsou například přepnutí mezi standardní a vysokou složitostí jednotlivých scénářů a možnost zapnout nebo vypnout obrázky na webovém rozhraní (což může ovlivnit dobu načítání jednotlivých stránek).

Poté, co jsme si nastavili požadovaný způsob ukládání dat, je potřeba ještě vygenerovat testovací data. Iniciální hodnoty jsou nastaveny na 200 uživatelských účtů a 400 akcií různých firem, ovšem tyto hodnoty mohou být libovolně změněny. Nyní už zbývá jen použít náš oblíbený nástroj na generování zátěže a zvolit typ testu, který chceme provést. Jak již bylo zmíněno, můžeme testovat jednotlivé komponenty, zajímá-li nás ale výkon serveru při zatížení více komponent, máme na výběr ze dvou možností. Můžeme použít skriptovací modul některého z testovacích nástrojů a připravit si zátěž přesně podle vlastních představ a nebo můžeme využít předem připraveného servletu. Pokaždé, když je zavolána adresa tohoto servletu, je provedeno přihlášení za náhodného uživatele a vykonání jednoho z nachystaných scénářů. Samotné vykonání scénáře věrně simuluje volání ostatních servletů potřebných k provedení daného scénáře společně se zadáváním náhodných hodnot. Tímto je zaručeno generování stejné zátěže jako při použití skriptovacího modulu. Generování zátěže pomocí tohoto servletu potom znamená pouze volání servletu virtuálními uživateli, o zbytek práce už se postará DayTrader sám.

\section{Ladění serveru}
Nyní, když víme, jaké výkonnosti chceme s naší aplikací dosáhnout, a máme k dispozici také aplikační server, na kterém naše aplikace poběží, můžeme spustit testy, abychom zjistili, jak si naše aplikace stojí oproti požadavkům. Máme-li štěstí, dosáhneme požadovaných hodnot z SLA napoprvé a aplikace nepotřebuje žádné ladění. Ovšem to se zase tak často nestává. Jednou z možností jak zlepšit výkon naší aplikace je optimalizace serveru, na kterém je naše aplikace nasazena. I přesto, že se většina expertů shodne na tom, že 75 \% výkonu aplikace závisí na tom, jak je aplikace implementována, špatně nakonfigurované běhové prostředí může negativně ovlivnit pocit uživatelů z práce s naší aplikací~\cite{performanceTuning}. 

U serveru nás nejvíce zajímají následující parametry:

\vspace{5 mm}\noindent{\bf Vytížení JVM}
\\\indent Stejně jako každá aplikace i aplikační server běží na JVM. Optimalizace JVM v zásadě znamená nastavení správné velikosti haldy a vybrání vhodného algoritmu pro Garbage collector~\footnote{mechanismus, pomocí kterého jsou odstraňovány z paměti JVM objekty, na které již neexistují žádné reference}.

\vspace{5 mm}\noindent{\bf Fond vláken}
\\\indent Každý nový požadavek na aplikaci je nejprve vložen do fronty. Tato fronta je obsluhována fondem vláken, které požadavek vyjme a zpracuje. Je-li fond vláken příliš veliký, dochází ke zbytečnému zabírání systémových zdrojů, které mohly být použity jinde, ba dokonce může dojít k degradaci výkonu v důsledku častého přepínání kontextu. Je-li naopak počet vláken příliš malý, trpí tím propustnost aplikace, protože žádná volná vlákna jsou ekvivalentem žádné práce navíc.

\vspace{5 mm}\noindent{\bf Fond JDBC připojení}
\\\indent Většina aplikací používá jako datové úložiště databázi. Každý požadavek na přístup k databázi je před vykonáním uložen do fondu připojení (connection pool). Nemá-li tento fond žádné volné připojení, musí požadavek čekat, což má negativní vliv na výkonnost aplikace.

\vspace{5 mm}\noindent{\bf Cache}
\\\indent K redukci počtu přístupů k databázi lze použít cache, protože poskytnout obsah vnitřní paměti bude vždy rychlejší než přistupovat k databázi. Ovšem je potřeba mít se na pozoru, protože enormní velikost cache může zabrat veškeré místo na haldě a způsobit selhání aplikace. Také správu cache nelze zanedbat. Při zaplnění cache a volání stále nových prvků se může stát, že aplikace stráví víc času řízením cache než poskytováním dat (thrashing). V tomto případě ztrácí cache veškeré své výhody. Je proto potřeba zvážit využití cache paměti v závislosti na povaze konkrétní aplikace. 

\vspace{5 mm}\noindent{\bf Fondy pro bezstavové komponenty}
\\\indent Veškeré bezstavové komponenty jsou uloženy ve fondu. Jakmile proces potřebuje zdroj, je mu poskytnut a následně vrácen do fondu. Bezstavovost nám umožňuje poskytnout procesu pokaždé jiný zdroj. Správně nastavená velikost fondu by měla zaručovat dostatek zdrojů pro všechny procesy, ovšem velikost fondu by neměla být zbytečně veliká z důvodu paměťových nákladů na údržbu.

\vspace{5 mm}\noindent{\bf JMS fond}
\\\indent K dosažení asynchronního zpracování používají aplikace zprávy (messaging). Tyto zprávy jsou na straně serveru zachycovány JMS fondem. JMS fond může mít (v závislosti na implementaci) nastaven maximální počet zpráv (bytů), který je schopen příjmat. Je-li JMS fond plný a pokusí se přijímat novou zprávu, může dojít k totálnímu selhání služby.













\chapter{Architektura aplikace}
V této kapitole se zaměříme na architekturu aplikace a společně s ukázkami si blíže představíme některé specifikace Java EE platformy, které byly při implementaci použity. Nejprve si krátce představíme novinky v Java EE 6 platformě a směr, kterým se celá platforma ubírá. Jednotné vrstvy aplikace budeme postupně procházet v kapitolách 4 - 6.

\section{Třívrstvý model}
V současné době je běžné, že kromě webových prohlížečů, je k aplikaci přistupováno také pomocí řady jiných klientských zařízení jako jsou PDA a mobilní telefony. Webová aplikace, běžící v prohlížeči nebo v mobilním zařízení, zobrazuje uživatelské rozhraní (formuláře, tlačítka, tabulky) za pomoci technologií jako JSP, JSF nebo Java sevletů. Typické akce uživatele (zadání vyhledávacích kritérií, přidání položky do košíku a jiné) jsou obsluhovány pomocí volání metod session bean, které běží v EJB kontejneru. Poté, co je session beana zavolána, zpracuje požadavek a vrátí ho webové aplikaci. Ta dá požadavku patřičný formát s pošle ho zpět na klientské zřízení.

\begin{figure}[!ht]
\centering
\includegraphics[width=0.8\textwidth]{tier}
\caption{Model třívrstvé architektury~\cite{kodali06}}
\label{img:tier}
\end{figure}

V třívrstvé architektuře mohou klientské aplikace (běžící ve webovém kontejneru) a session beany (běžící v EJB kontejneru) sdílet stejné prostředí (aplikační server), mohou ovšem také běžet na separátních strojích.

Jako vhodný typ aplikace pro výkonnostní testování byla vybrána zjednodušená verze internetového obchodu. Základní scénář pro generování zátěže je následující. Zákazník se přihlásí do aplikace pod svým uživatelským jménem, vybrané produkty z katalogu umístí do košíku a jakmile je se svým výběrem spokojen, tak objednávku dokončí. Při dokončení objednávky je zboží vybrané zákazníkem zarezervováno na skladě. Nyní přichází na řadu skladník, který sebere objednávku, odečte objednané kusy zboží ze skladu a vystaví fakturu. Skladník se stará také o to, abychom měli na skladě neustále dostatek zboží. Pro snadnější dodržování architektury je aplikace tvořena 2 moduly, které spolu vzájemně spolupracují: Web modul a EJB modul (viz Obrázek~\ref{img:overview}).

\begin{figure}[!ht]
\centering
\includegraphics[width=0.8\textwidth]{overview}
\caption{Vysokoúrovňový přehled architektury aplikace}
\label{img:overview}
\end{figure}

\begin{itemize}
  \item prezentační vrstva aplikace je vytvořena za pomocí specifikace JSF a knihovny komponent PrimeFaces
  \item s aplikací je možné komunikovat také pomocí rozhraní REST
  \item komunikaci mezi prezentační vrstvou a aplikační vrstvou mají na starosti Managed beany s podporou specifikace CDI
  \item aplikační vrstva je tvořena pomocí specifikace EJB 3.1.
  \item persistenci dat má na starosti specifikace JPA 2.0
  \item v aplikaci nalezneme také skladníka, který je implementován pomocí specifikace JMS a speciální časovanou službu ve třídě Controller
\end{itemize}

\section{Java EE 6 platforma}
Jedním ze základních požadavků na aplikaci bylo, aby běžela na platformě Java EE 6, díky čemuž se zajistí její přenositelnost na všechny ostatní servery s certifikací na tuto platformu. Java EE 6 pokračuje v rozvíjení trendů zavedených v předchozí verzi a to jsou zejména odlehčení a zjednodušení celého modelu. V současné době musí aplikační server podporovat 28 různých specifikací, aby mohl být certifikovaný jako Java EE 6 server. Aby nedocházelo k dalšímu nárůstu, je velikost platformy redukována pomocí profilů (viz~\ref{profily}) a prořezávání. 
\subsection{Prořezávání}
\label{pruning}
Proces prořezávání se používá již od roku 1999. Se stále rostoucím počtem specifikací, které byly do Java EE platformy přidávány, rostla také její velikost. Postupem času se začaly objevovat specifikace, které nezaznamenaly takový úspěch, jaký se od nich očekávalo, a nebo se staly technicky zastaralými, proto bylo rozhodnuto o jejich odstranění. Odstraněním se myslí buď jejich nahrazení novější specifikací a nebo prosté vypuštění z celé platformy (specifikace se poté může dál samostatně vyvíjet i mimo platformu).

Mezi specifikace, které budou v příští verzi Java EE platformy vypuštěny, patří:
\begin{itemize}
  \item {\bf EJB 2.x Entity Beany (částí JSR 318)} - komplexní a těžkopádné komponenty pro persistenci dat, již dnes jsou nahrazeny JPA
  \item {\bf JAX-RPC (JSR 101)} - první pokus o modelování SOAP webových služeb jako volání RPC, nahrazeno JAX-WS
  \item {\bf JAXR (JSR 93)} - aplikační rozhraní pro komunikaci s UDDI registry. Pro nevalný úspěch bude odstraněno. Stejný osud potká i Java EE Application Deployment (JSR 88) a Java EE Management (JSR 77)
\end{itemize}

\subsection{Anotace}
Ke zjednodušení přispělo představení anotací v Java EE 5, které poskytují metadata potřebná serverem pro určení chování jednotlivých komponent. Před představením anotací byly XML dokumenty obsahující konfigurace jednotlivých komponent jedinou možností, jak serveru tyto metadata poskytnout. Snižující se počet komponent vyžadujících XML dokumenty pozitivně ovlivnil rychlost a jednoduchost vývoje. Ovšem tato změna postihla v Java EE 5 pouze EJB, entity a webové služby, kdežto prezentační vrstva aplikace z těchto změn zatím těžit nedokázala. V Java EE 6 se stalo používání XML dokumentů volitelným i pro prezentační vrstvu a veškeré funkcionality lze nyní dosáhnout pomocí anotací (podpora pro XML dokumentů zůstala nezměněná).


\subsection{EJB}
Další z komponent, které prošly významnou změnou jsou EJB. EJB jsou komponenty na straně serveru, které zastřešují aplikační logiku a starají se o transakce, zabezpečení, plánování, vzdálený přístup, webové služby a jiné. Tyto již nemusí implementovat žádné rozhraní, ale stačí obyčejná třída s anotací označující typ beany. Ve verzi Java EE 6 máme na výběr z těchto typů~\footnote{EJB mohou být také použity jako koncové body webových služeb jako REST a SOAP}

\begin{itemize}
  \item {\bf Bezstavová beana} - klient může použít jakoukoliv instanci beze změny výsledku
  \item {\bf Stavová beana} - uchovává si stav napříč celým sezením konkrétního klienta
  \item {\bf Singleton} - jediná instance, která je sdílená napříč aplikací
  \item {\bf Message driven beana} - použity pro integraci s externími systémy pomocí JMS
  \item {\bf Entity beana} -  používány pro persistenci dat (označeny k prořezání, viz~\ref{pruning})
\end{itemize}

Bezstavové beany, stavové beany a Singleton sborně nazýváme Session beanami.

Jelikož v předchozí verzi specifikace jasně nedefinovala konvence pro tvorbu JNDI~\footnote{Java Naming and Directory Interface - rozhraní, které umožňuje klientům hledat objekty pomocí jména} jmen, bylo běžné, že různé aplikační servery přiřazovaly EJB různá jména. Proto bylo potřeba měnit kód v závislosti na tom, který server jsme právě používali. Tento problém je již minulostí, jelikož v Java EE 6 jsou konvence pro tvorbu JNDI jmen objektů přímou součástí specifikace.

\subsection{CDI}
Je specifikace popisující správu životního cyklu kontextu a injektování závislostí, která přebírá to nejlepší z nástaveb jako Spring, Seam nebo Guice. Tato specifikace definuje sadu komplementárních služeb, které zlepšují strukturu aplikačního kódu. Od Java EE 6 se stala součástí platformy, tudíž musí být podporována všemi servery. CDI nám umožňuje:

\begin{itemize}
  \item typově bezpečný přístup k injektování závislosti
  \item interakci objektů pomocí událostí
  \item vylepšení životního cyklu stavových objektů
  \item volání bean na JSF stránkách pomocí sjednoceného jazyka výrazů (EL)
\end{itemize}

Každý klient vidí beanu v jiném stavu, což závisí pouze na tom, na kterou instanci má právě odkaz. Nicméně pro bezstavové beany a singletony platí, že klient nemá žádnou kontrolu nad jejich životním cyklem. Každá beana má definovaný rozsah, který určuje:

\begin{itemize}
  \item životní cyklus každé instance dané beany 
  \item klienty, kteří mohou sdílet odkaz na konkrétní instanci dané beany
\end{itemize}

CDI definuje čtyři vestavěné rozsahy:

\begin{itemize}
  \item{\bf @RequestScoped} - kontext je unikátní v rámci daného požadavku
  \item{\bf @SessionScoped} - kontext je unikátní v rámci daného sezení
  \item{\bf @ApplicationScoped} - kontext je unikátní v rámci dané aplikace
  \item{\bf @ConversationScoped} - speciální verze @SessionScoped
\end{itemize}

Pro libovolné vlákno v aplikaci využívající CDI může existovat kontext, který je pro toto vlákno unikátní (má-li beana nastavený rozsah na požadavek), nebo může být sdílený s určitou skupinou vláken (má-li beana nastavený rozsah na sezení) a nebo může být sdílený všemi vlákny (má-li beana nastavený rozsah na aplikaci). CDI nám umožňuje za běhu aplikace změnit jednu beanu za jinou implementující stejné rozhraní, avšak s různým životním cyklem (rozsahem)~\cite{weld}.

CDI se výborně hodí pro práci se zdroji aplikačního serveru. Anotace @Produces nám umožňuje definovat zdroje na jednom místě. Pomocí CDI máme možnost k těmto zdrojům přistupovat napříč celou aplikací uniformním způsobem.

\begin{lstlisting}
public class Resources {
	@Produces
	public Logger produceLog(InjectionPoint injectionPoint) {
		return Logger.getLogger(injectionPoint.getMember()
				.getDeclaringClass().getName());
	}
}
\end{lstlisting}

Pomocí anotace @Inject získáme nejenom instanci třídy Logger, ale tato instance bude rovnou serverem naplněna patřičnými parametry, které server získá z bodu injektace.

Anotace @Produces nám dále umožňuje použít návratovou hodnotu metody přímo v našich JSF stránkách. Mějme metodu s touto anotací vracející seznam všech zákazníků:

\begin{lstlisting}
@Produces
@Named
List<Customer> getCustomerList() {
	return customerList;
}
\end{lstlisting}

Vypsání zákazníku na našich stránkách provedeme prostým:

\begin{lstlisting}
<h:dataTable value="#{customerList}" var="_customer">
	<h:column>
	...
	</h:column>
</h:dataTable>
\end{lstlisting}

Aby bylo možné v naší aplikaci používat CDI, je nutné aby byl přítomen soubor beans.xml. Tento soubor může být prázdný, ale musí existovat. U webových aplikací musí být přítomen v adresáři WEB-INF, u EJB modulů a Java archívů potom v adresář META-INF.

\subsection{Validace Bean}
Je další z řady novinek v Java EE 6. Poskytuje nám jednotný směr k definování omezení na objektový model, čímž dokážeme zajistit integritu dat v databázi~\footnote{podobný princip validace umožňuje také JSF, kdy jsou data validována na prezentační vrstvou}. Omezení definujeme uvnitř entit pomocí anotací například následujícím způsobem:

\begin{lstlisting}
@Entity
public class Product implements Serializable {
	@Id
	private Long id;
	@Size(min = 1, max = 25, message = "Name may not be null")
	@Pattern(regexp = "[A-Za-z0-9]*", 
		message = "Name must contain only letters and numbers")
	private String name;
	... metody get a set
}
\end{lstlisting}

Takto definovaná omezení jsou použita naším poskytovatelem JPA při ukládání nebo změně dat v databázi. Pokud by transakce jakékoliv z omezení porušila, došlo by k jejímu zrušení a vrácení do původního stavu (rollback). Těchto omezení se však dá využít i na ostatních vrstvách aplikace. V naší aplikace je použit k validaci dat následující generický validátor, který umožní validovat všechny entity, které se v aplikaci vyskytují. Tento validátor může být následně použit na všech vrstvách aplikace.

\begin{lstlisting}
public class EntityValidator<T> {
	public Set<ConstraintViolation<T>> validate(T entity) {
		ValidatorFactory factory = Validation.buildDefaultValidatorFactory();
		Validator validator = factory.getValidator();
		Set<ConstraintViolation<T>> constraintViolations = validator.validate(entity);
		return constraintViolations;
	}
}
\end{lstlisting}

Díky tomuto jsme schopni validovat data například již při čtení z formuláře a zamezit tak zbytečnému přístupu k databázi. Zprávy popisující omezení, která jsme porušili můžeme rovnou použít jako výstup pro uživatele (viz~\ref{jsfMessage})

\subsection{REST}
S rostoucí popularitou webových služeb na přístupu REST~\footnote{Representational state transfer - architektura založená na webových standardech a HTTP protokolu (viz~\ref{rest})} byla do Java EE 6 přidána specifikace JAX-RS. Stačí jednoduchá třída se speciální anotací, která umožní komunikovat s naší aplikací pomocí HTTP protokolu.

\chapter{Persistentní vrstva}
Existuje více způsobů, kterými můžeme v jazyce Java ukládat stav objektů. Můžeme například použít mechanismus serializace (převedení objektů na sekvenci bitů). Objekt může být takto uložen na disk, nebo poslán pomocí internetu a použit na jiném stroji. Serializované objekty si samozřejmě zachovávají platformovou nezávislost. Java poskytuje jednoduchý a transparentní mechanismus na serializaci objektů. Stačí, bude-li třída, jejíž instanci chceme serializovat, implementovat rozhraní java.io.Serializable. Jelikož se jedná o velice jednoduchý mechanismu, není k dispozici žádný dotazovací jazyk ani infrastruktura, která by podporovala sdílený přístup k datům~\cite{goncalves09}.

Dalším způsobem, kterým můžeme data ukládat je za použití JDBC, které je součástí Java platformy už od verze 1.1. JDBC nám umožní připojit se k databázi a dotazovat se na ni pomocí SQL. Ačkoliv je tento přístup stále široce používán, je čím dál více zastíněn objektově relačním mapováním (ORM). U ORM jsou Java třídy mapovány na tabulky relační databáze, čímž vzniká dojem, že nepracujeme s relační databází, ale databází objektů.
JPA 2.0

JPA je abstrakcí nad JDBC díky, které jsme schopni se úplně osvobodit od SQL. Mezi hlavní komponenty JPA patří:

\begin{itemize}
  \item{\bf ORM} - mechanismus mapování entit na data uložená v relační databázi 
  \item{\bf aplikační rozhraní manažera entit} - umožňuje nám volat CRUD operace bez přímého použití JDBC
  \item{\bf JPQL} - objektově orientovaný dotazovací jazyk
\end{itemize}

\section{ORM}
Jakmile se začínáme bavit o mapování objektů do relační databáze, je na místě začít používat pojem entita namísto pojmu objekt. Objekty nazýváme instance určité třídy, které zrovna existují v paměti. Entity jsou objekty se schopností být namapovány do databáze, které existovaly krátce v paměti, a nyní jsou uchovány v databázi. Jakmile je entita jednou namapována, tak může být spravována pomocí JPA (můžeme ji uložit, změnit, vymazat a dotazovat se na ní pomocí JPQL).

\subsection{Entitní třídy}
Jak tedy proměnit třídu v třídu entit? Aby se z obyčejné třídy stala entitní třída, tak musí splňovat následující:
\begin{itemize}
  \item třída musí mít anotaci @Entity
  \item anotace @Id musí být použita k určení primárního klíče
  \item třída musí mít bezparametrický konstruktor s veřejným nebo chráněným modifikátorem přístupu (mimo to může mít i další konstruktory)
  \item výčtová třída nebo rozhraní nemůže být entitní třídou
  \item bude-li k této třídě přistupováno pomocí vzdáleného rozhraní, tak musí implementovat rozhraní Serializable
  \item entitní třída ani žádné její atributy nesmějí být deklarovány jako finální
\end{itemize}

Následující příklad ukazuje nejjednodušší možnou entitní třídu.

\begin{lstlisting}
@Entity
public class Product {
	@Id
	private Long id;
	private String name;
	private Float price;
	public Product() {}
	... metody get a set
}
\end{lstlisting}

Jelikož jsou splněna všechna výše uvedená pravidla, můžeme třídu Product nyní považovat za entitní třídu. V JPA se stejně tak jako v ostatních specifikacích Java EE razí přístup konvence nad konfigurací, což znamená, že nedefinujeme-li jinak, tak mají být použity standardní hodnoty. Toto nám umožňuje psát minimum kódu navíc, jelikož se můžeme spolehnout na standardy. Ponecháme-li třídu Product beze změny, bude namapována například způsobem, který můžeme vidět na obrázku~\ref{img:orm} (hodnoty se mohou měnit v závislosti na použité databázi).

\begin{figure}[!ht]
\centering
\includegraphics[width=0.7\textwidth]{EntityToTableMapping}
\caption{ORM v praxi}
\label{img:orm}
\end{figure}

\subsection{Konfigurovatelnost pomocí anotací}
Jestliže dojde k situaci, že se název tabulky nebo sloupce shoduje s některým z klíčových slov SQL (například Order), nebo chceme z nějakého důvodu způsob mapování změnit (například budujeme aplikaci nad existující databází), máme možnost takto učinit pomocí anotací (případně pomocí XML dokumentu). Stačí prostudovat aplikační rozhraní dané anotace a upravit si mapovaní dle vlastních představ. Na ukázce vidíme aplikační rozhraní anotace @Column, kterou lze použít pro úpravu způsobu mapování jednotlivých atributů na sloupce.

\begin{lstlisting}
@Target({METHOD, FIELD}) @Retention(RUNTIME)
public @interface Column {
String name() default "";
boolean unique() default false;
boolean nullable() default true;
boolean insertable() default true;
boolean updatable() default true;
String columnDefinition() default "";
String table() default "";
int length() default 255;
int precision() default 0;
int scale() default 0; 
}
\end{lstlisting}

Dalšími z užitečných anotací v JPA jsou například:

\begin{itemize}
  \item{\bf @Transient} - jakmile označíme třídu anotací @Entity, tak jsou všechny atributy automaticky mapovány do tabulky. Chceme-li nějaký atribut vyloučit, můžeme použít anotaci @Transient
  \item{\bf @Temporal} - v Javě můžeme použít pro uložení časových dat třídy java.util.Date a java.util.Calendar. Pro použití těchto tříd v ORM slouží anotace @Temporal
  \item{\bf @Enumerated} - chceme-li použít jako atribut výčtovou třídu
\end{itemize}

\subsection{Vazby mezi entitami}

Mezi jednotlivými třídami můžeme definovat několik typů vazeb, které nám určují vztahy mezi nimi. Každá vazba definuje směr. Může být buď jednosměrná (můžeme se z jednoho objektu dostat na druhý, ale ne naopak), nebo obousměrná (objekty se vidí navzájem). Další vlastností vazeb je jejich kardinalita, která určuje, kolik objektů je ve vazbě zahrnuto. Každý vztah má vlastníka, v případě, že je vazba jednosměrná, je vlastník implicitně daný, je-li vazba obousměrná, musí být explicitně určen. 

V JPA jsme schopni upravovat vztahy mezi entitami pomocí anotací @OneToOne, @OneToMany, @ManyToOne a @ManyToMany. Pomocí těchto anotací jsme schopni definovat směr vazby i způsob uložení dat v tabulkách (má-li být použit sloupec s cizím klíčem nebo sjednocená tabulka).

Pomocí atributu fetch můžeme u každé z těchto vazeb nastavit způsob načítání dat z tabulky, ten může být buď líný, nebo horlivý. Mějme například entitu zákazníka. Každý zákazník může mít historii svých objednávek. Každá objednávka se může skládat z několika položek s různými produkty. Za podmínky, že bychom měli nastavené horlivé načítání u všech entit, načetly by se do paměti při vyhledání zákazníka v databázi také všechny jeho objednávky a všechny produkty a to i v případě, kdy nás zajímalo pouze zákazníkovo jméno. Takové chování by u aplikace znamenalo značnou degradaci její výkonnosti. Anotace @OneToOne a @ManyToMany používají standardně horlivé načítání, anotace @OneToMany a @ManyToMany líné. K tomu, abychom donutili JPA líně načíst data, můžeme použít metodu initialize, kterou nabízí náš JPA poskytovatel Hibernate. Po zavolání:
\begin{lstlisting}
Hibernate.initialize(order.getOrderItems());
\end{lstlisting}
	můžeme pracovat i s jednotlivými instancemi položek objednávky.

Aby zůstala naše data konzistentní, je vhodné odstraňovat z tabulek řádky, na které není odkazováno z žádné jiné tabulky. V případě, že bychom například smazali z databáze zákazníka, seznam jeho objednávek by pro nás ztratil význam. Pomocí atributu orphanRemoval upozorníme JPA, aby po smazání zákazníka smazalo také všechny jeho objednávky.
\begin{lstlisting}
...
	@OneToMany(mappedBy = "customer", orphanRemoval = true)
	private List<Order> order;
...
\end{lstlisting}
\subsection{Dědičnost v JPA}
JPA podporuje tři různé strategie při mapování dědičnosti entit. Tato strategie je vždy definována ve třídě s nejvyšší hierarchií pomocí anotace @Inheritance.

\vspace{5 mm}\noindent{\bf Strategie jedné tabulky}
\\\indent V případě, že anotace @Inheritance chybí, je vždy použita tato strategie. Všechny entity, nehledě na postavení v hierarchii, jsou ukládány do jedné tabulky. Ačkoli je tato strategie nejjednodušší, má několik nevýhod. V tabulce vznikají prázdná místa, protože různé třídy mají různé atributy, a ani přidání nové entity do hierarchie se neobejde bez problémů, protože je potřeba přidat nové sloupce do tabulky. Navíc je potřeba, aby atributy všech entit kromě rodičovské měly povoleny prázdné hodnoty, jinak nebude možné do tabulky vkládat.

\vspace{5 mm}\noindent{\bf Strategie sjednocené tabulky}
\\\indent Každá entita má vlastní tabulku, ale společné atributy entit jsou uloženy pouze do rodičovské tabulky. Ta obsahuje extra sloupec označující entitu, které daný řádek patří. Ani tato strategie není zrovna ideální, protože musíme často spojovat tabulky, abychom dostali úplný stav entity.

\vspace{5 mm}\noindent{\bf Strategie jedné tabulky pro každou třídu}
\\\indent Podobně jako u strategie sjednocené tabulky i zde má každá entita vlastní tabulku s tím rozdílem, že všechny atributy rodiče budou mapovány také v tabulce potomka. Při použití této strategie nenajdeme žádné sjednocené sloupce ani hodnoty. Nevýhodnou se tato strategie stává, chceme-li najít všechny záznamy rodiče i všech synů, protože v tomto případě se musí provést sjednocení výsledků z tabulek všech entit.

Nyní, když známe jednotlivé strategie dědičnosti, je nutné uvést, že entity nemusí nutně dědit zase jenom z entit. Existují tři další typy tříd, od kterých mohou entity dědit:

\vspace{5 mm}\noindent{\bf Abstraktní entita}
\\\indent Z pohledu JPA se vůbec nic nemění, je možné definovat všechny tři strategie úplně stejně, jako by se jednalo o normální entitu, jediný rozdíl je v tom, že abstraktní entita nemůže být instancializována (stejne tak jako žádná jiná abstraktní třída v jazyce Java).

\vspace{5 mm}\noindent{\bf Neentita}
\\\indent Dědí-li entita z neentity, jsou při jejím uložení brány v potaz pouze její vlastní atributy. Jakýkoliv stav rodičovské neentity je ztracen.

\vspace{5 mm}\noindent{\bf Mapovaná supertřída}
\\\indent Tento speciální druh třídy se používá nejen pro sdílení stavu a chování, ale také pro sdílení informací o mapování pro entity, které od této třídy dědí. Tyto třídy nejsou nijak spravovány JPA poskytovatelem ani nemají vlastní tabulku. Ve vyvíjené aplikaci je použita mapovaná supertřída OrderRoot, která sdružuje společné vlastnosti tříd Order a Invoice, které z této třídy dědí.

\begin{lstlisting}
@MappedSuperclass
@Inheritance(strategy = InheritanceType.TABLE_PER_CLASS)
public class OrderRoot {
	@Id
	@GeneratedValue(strategy = GenerationType.AUTO)
	private Long id;
	@JoinColumn(nullable = false)
	@ManyToOne
	private Customer customer;
	@Temporal(TemporalType.TIMESTAMP)
	private Date creationDate;
...
}
\end{lstlisting}

\section{Manažer entit}

JPA nám neposkytuje pouze možnost entity mapovat, ale také s nimi pracovat a dotazovat se na ně. K tomuto účelu v JPA slouží manažer entit, který nám poskytuje rozhraní pro ukládání, vyhledávání, mazání a synchronizaci objektů s databází. Dovoluje nám dotazovat se na databázi pomocí JPQL~\footnote{Java Persisten Query Language - jazyk pro práci s databází objektově orientovaným způsobem}. JPQL nemá ponětí o struktuře dat uložených v databázi ani o řádcích a sloupcích, místo toho nám dovoluje přistupovat k datům jako k objektům s atributy pomocí standardní tečkové notace. Manažer entit je centrálním bodem v JPA. Má na starosti životní cykly entit, vyhledávání entit pomocí klíče, může zamykat přístup k entitám, aby zajistil konzistenci dat při sdíleném přístupu, a nebo nám vracet hodnoty JPQL dotazů.

Použití manažeru entit se zásadně liší podle toho, jestli se jedná o aplikaci běžící na serveru (transakce, životní cykly entit a jiné jsou spravovány serverem) nebo Java SE aplikaci, kde se musí o vše postarat aplikace sama.

Použijeme-li již známou třídu Resources, může vytvoření nového produktu v aplikaci spravované serverem vypadat následovně:

\begin{lstlisting}
@Stateless
public class ProductManager {
	@Inject
	private EntityManager em;
	public Product addProduct(String name, Long price) {
		Product product = new Product(name, price);
		em.persist(product);
		return product;
	}
}
\end{lstlisting}

Nemusíme manažer entit vytvářet, ani zavírat, jelikož je jeho životní cyklus spravován serverem. Ke konfiguraci manažeru entit slouží persistentní jednotka, která je definována v souboru persistence.xml v adresáři META-INF naší aplikace. Požadavky na tento soubor se opět liší vzhledem k tomu, zda-li aplikace poběží na serveru nebo ne. Tento soubor by měl obsahovat adresu databáze, do které chceme data ukládat a různá nastavení, kterými řekneme našemu poskytovateli JPA, jaké chování od něj očekáváme (strategie vytváření tabulek, logování SQL dotazu a jiné).

\section{JPQL}

V naší aplikaci si většinou nevystačíme s pouhým ukládáním, mazáním a vyhledáváním pomocí klíče, ale často potřebujeme vyhledat data pomocí specifických kritérií. Zde přichází na řadu JPQL. JPQL do jisté míry vychází ze SQL s tím rozdílem, že výsledkem volání SQL jsou řádky a sloupce, kdežto výsledkem volání JPQL jsou entity nebo kolekce entit. Syntaxe JPQL je velice bohatá a umožňuje nám budovat i složité dotazy nebo dotazy s parametry. 

V JPA 2.0 máme čtyři různé druhy dotazů, které můžeme volat:
\begin{itemize}
  \item {\bf dynamický dotaz} - nejjednodušší forma použití JPQL, dotaz je specifikován za běhu aplikace ve formě řetězce
  \item {\bf pojmenovaný dotaz} - pojmenované dotazy jsou statické a neměnné (mohou obsahovat parametry)
  \item {\bf nativní dotaz} - v tomto typu dotazu používáme přímo SQL namísto JPQL
  \item {\bf kritéria API} - zcela nový druh typovaných dotazu v JPA 2.0
\end{itemize}

Dotazy vytvořené pomocí kritéria API mají oproti JPQL několik zásadních výhod. Jsou typově zabezpečené, tudíž nám poskytují typovou kontrolu již v době kompilace kódu a nejsou náchylné k překlepům, jelikož lze využít doplňování kódu během tvorby našich dotazů, ovšem oproti JPQL je psaní těchto dotazů nepříjemně těžkopádné. Zatímco získání jmen všech produktů z databáze pomocí kritéria API vypadá následovně:

\begin{lstlisting}
Metamodel mm = em.getMetamodel();
EntityType<Product> mproduct = mm.entity(Product.class);
SingularAttribute<Product, String> name = mproduct
	.getDeclaredSingularAttribute("name", String.class);
CriteriaBuilder cb = em.getCriteriaBuilder();
CriteriaQuery<String> criteria = cb.createQuery(String.class);
Root<Product> product = criteria.from(Product.class);
criteria.select(product.get(name));
return em.createQuery(criteria).getResultList();
\end{lstlisting}

Stejná věc by šla udělat pomocí JPQL následujícím dotazem:

\begin{lstlisting}
Query query = em.createQuery("SELECT p.name FROM Product p");
return query.getResultList();
\end{lstlisting}

Z důvodu typové bezpečnosti bylo v aplikaci použito výhradně nového kriteria API.

\section{Datový model aplikace}
Základ aplikace tvoří entity Customer a Product. Pomocí atributu orphanRemoval specifikovaného u jednotlivých vazeb s těmito entitami jsme schopni smazat všechna data aplikace pouhým smazáním obsahů tabulek zákazníků a produktů, čehož později využijeme k získání konzistentních podmínek při opakování testů. Vztahy mezi třídami jsou patrné z následujícího diagramu tříd (viz obrázek~\ref{img:class}).

\begin{figure}[!ht]
\centering
\includegraphics[width=0.7\textwidth]{class}
\caption{Diagram tříd}
\label{img:class}
\end{figure}

V modelu byla využita mapovaná supertřída OrderRoot, ze které dědí třídy Order a Invoice. Na ERD diagramu (viz obrázek~\ref{img:erd}) můžeme vidět, jakým způsobem jsou atributy definované v mapované supertřídě přeneseny do jejich potomků

\begin{figure}[!ht]
\centering
\includegraphics[width=0.9\textwidth]{erd_diagram}
\caption{Entitně relační diagram}
\label{img:erd}
\end{figure}


\chapter{Aplikační vrstva}
Entity by měly sloužit pouze jako datový model aplikace a neměly by zahrnovat žádnou složitější aplikační logiku. Stejně tak bychom neměli míchat aplikační logiku do uživatelského rozhraní.  Kdyby naše aplikace poskytovala více uživatelských rozhraní (web, přenosná zařízení a jiné), museli bychom psát veškerou funkcionalitu pro každé z nich zvlášť. K rozdělení těchto dvou vrstev nám slouží aplikační vrstva. Aplikační vrstva nám poskytuje funkcionalitu aplikace a modeluje akce. Aplikační vrstva slouží jako vstupní bod pro klientská rozhraní, zajišťuje zabezpečení aplikace, správu transakcí, komunikaci s externími webovými službami, zasílání asynchronních zpráv pomocí JMS a mnoho dalších~\cite{goncalves09}. 

V Java EE je tato vrstva implementována pomocí EJB. EJB jsou podobně jako entity tvořeny obyčejnými třídami doplněnými anotacemi. S verzí 3.1 se z EJB staly odlehčené komponenty podporující přístup konvence nad konfigurací. Programátor se může soustředit výhradně na tvorbu funkcionality. Jakmile je EJB nasazena,  server ji poskytuje následující služby~\cite{wikiEJB}:

\begin{itemize}
  \item {\bf komunikace se vzdáleným klientem} - zjednodušuje komunikaci mezi klientem a aplikací
  \item {\bf injektování závislosti} - zajišťuje naplnění deklarovaných proměnných (datových atributů) např. dalšími EJB, JMS, datovými zdroji (manažer entit), a jiné
  \item {\bf řízení stavu} - kontejner udržuje v paměti stavy jednotlivých stavových bean a tím i vzdálený stav u klienta, kterému se jeví stav jakoby uložen lokálně
  \item {\bf poskytování fondů} - vytváření fondů instancí pro bezstavové beany a message-driven beany (MDB)
  \item {\bf řízení životního cyklu} - stará se o vytváření, inicializaci a destrukci instancí bean
  \item {\bf messaging} - umožňuje MDB poslouchat na JMS~\footnote{Java Messiging Service - rozhraní umožňující aplikacím asynchronně vytvářet, číst, posílat a příjímat zprávy} destinacích a konzumovat zprávy
  \item {\bf management transakcí} - beany deklarují transakční vlastnosti metod, kontejner řeší potvrzení a zrušení transakce
  \item {\bf bezpečnost} - deklarace přístupů na úrovni tříd a metod
  \item {\bf podpora souběžného zpracování} - vývojář nemusí řešit problémy synchronizace souběžných přístupů ke sdíleným datům
  \item {\bf správa interceptorů} - komponenty umožňující odchytávat okamžik před a po volání metody
  \item {\bf asynchronní volání metod} 
\end{itemize}

Instanci EJB můžeme v aplikaci získat buď pomocí JNDI, nebo pomocí injektování závislosti.
Existuje několik typů EJB, ale my se budeme věnovat pouze těm, které byly použity v naší aplikaci.

\vspace{5 mm}\noindent{\bf Bezstavové beany}
\\\indent Jsou ideální pro implementaci úkolů, které mohou být dokončeny voláním pouze jedné metody. Jsou sdružovány ve fondech, ze kterých jsou odebírány a po obsloužení požadavku zase vráceny zpět. Neuchovávají si žádný stav relevantní pro klienta mezi obsluhou jeho jednotlivých požadavků. Jsou přirozeně vláknově bezpečné. Server nezaručuje, že stejný klient vždy dostane stejnou instanci beany. Chceme-li ze třídy udělat bezstavovou EJB, musíme jí označit anotací @Stateless. V naší aplikaci jsou bezstavové beany použity jako základní stavební kámen pro veškerou aplikační logiku.

\vspace{5 mm}\noindent{\bf Singleton}
\\\indent Singletonem označujeme beanu, která je v celé aplikaci instanciována pouze jednou a má globálně sdílený stav mezi klienty (pracujeme-li v prostředí klastrů, má každý klastr svoji vlastní instanci singletonu).

\vspace{5 mm}\noindent{\bf MDB (Message-Driven Beany)}
\\\indent slouží k obsluze operací, které nevyžadují okamžitou odpověď tím, že asynchronně konzumují JMS zprávy. I když se chovají podobně jako bezstavové beany, klientské aplikace k nim nemohou přistupovat přímo, ale pouze zasíláním zpráv na adresu, na které MDB naslouchá.


\section{Management transakcí}

Management transakcí
Transakce slouží k zajištění konzistence dat i při souběžném přístupu více aplikací. Transakce logicky sdružuje větší počet jednoduchých operací. Tyto operace mohou být prováděny sekvenčně i paralelně. Jakmile jedna část transakce selže (například vyhodí výjimku), je celá transakce zrušena a data jsou vrácena do původního stavu. Každá transakce musí mít následující vlastnosti:

\begin{itemize}
  \item {\bf atomicita} - transakce je sice tvořena více operacemi, ale navenek se chová pouze jako jedna opera, která proběhne buď celá, nebo je celá zrušena
  \item {\bf konzistence} - po dokončení transakce musí být data v konzistentním stavu
  \item {\bf izolovanost} - externí aplikace nevidí vnitřní stav transakce
  \item {\bf trvalost} - jakmile transakce úspěšně proběhne, jsou změny uložené v databázi viditelné i pro ostatní aplikace
\end{itemize}

Stejně jako tomu bylo u JPA i zde se liší použití transakcí u aplikace, která běží na serveru od použití v aplikaci Java SE. Používáme-li transakce spravované serverem, nemusíme se starat o vyznačování hranic transakce, ani explicitně rušit transakci při zachycení výjimky. Samozřejmě i zde máme možnost chování transakcí plně řídit pomocí anotací. V případě, že nám standardní chování nevyhovuje, můžeme tak učinit pomocí anotace @TransactionalAttribetu. Této můžeme předat jednu z následujících hodnot:

\begin{itemize}
  \item {\bf REQUIRED} - Použito implicitně. Metoda musí být vždy zavolána uvnitř transakce. Je-li metoda zavolána klientem bez transakčního kontextu, je serverem vytvořen nový.
  \item {\bf REQUIRES\_NEW} - Server vždy vytvoří novou transakci. V případě, že je metoda volána v probíhající transakci, je tato transakce pozastavena na úkor nově vytvořené transakce. Poté, co nová transakce skončí, pokračuje původní transakce. Úspěch nebo neúspěch nové transakce nemá vliv na dokončení transakce původní.
  \item {\bf SUPPORTS} - Je-li transakční kontext klienta k dispozici, je metodou použit, v opačném případě je metoda zavolána bez transakčního kontextu. Vhodné pro přístup k datům určeným pouze pro čtení.
  \item {\bf MANDATORY} - Metoda vyžaduje aby byla spuštěna klientem s transakčním kontextem. Za podmínky, že klient tento kontext nemá, je vyhozena výjimka.
  \item {\bf NOT\_SUPPORTED} - Metoda nemůže být zavolána uvnitř transakčního kontextu klienta. Jestliže klient nějaký kontext vlastní, je probíhající transakce pozastavena a je spuštěna opět po dokončení metody.
  \item {\bf NEVER} - Metoda nesmí být zavolána uvnitř transakčního kontextu klienta. V případě, že klient nějaký kontext vlastní, je vyhozena výjimka.
\end{itemize}
Jedna z věcí, na kterou si musíme dát při konfiguraci transakcí pozor, je volání metod uvnitř jedné třídy. Mějme dvě metody uvnitř třídy OrderManager.

\begin{lstlisting}
public Order addOrderWithMap(String email, Map<Long, Long> productsWithQuantity) {
	List<OrderItem> orderItems = new ArrayList<OrderItem>();
	...
	return addOrder(email, orderItems);
}
\end{lstlisting}

a

\begin{lstlisting}
@TransactionAttribute(TransactionAttributeType.REQUIRES_NEW)
public Order addOrder(String email, List<OrderItem> orderItems) {
Order order = new Order();
	...
	return order;
}
\end{lstlisting}

Obě metody vrací instanci třídy Order. První metoda neudělá nic jiného, než že zapouzdří mapu čísel do seznamu položek objednávky a zavolá druhou metodu, která odvede veškerou práci. Problém by mohl nastat v momentě, kdy zavoláme první metodu uvnitř probíhající transakce. Jelikož nemá první metoda žádnou anotaci, je aplikován standardní přístup a metoda běží uvnitř transakčního kontextu klienta. Bohužel, dle specifikace i druhá volaná metoda převezme toto chování a anotace @TransactionAttribute je ignorována, tudíž i tato metoda proběhne uvnitř transakčního kontextu klienta. Naštěstí existuje způsob, jak toto chování obejít. Pomocí injektování závislosti získáme odkaz na jinou instanci této třídy, která vykoná druhou metodu s patřičným nastavením transakce. První metoda bude nyní vypadat takto:

\begin{lstlisting}
@EJB
private OrderManager manager;
public Order addOrderWithMap(String email, Map<Long, Long> productsWithQuantity) {
	List<OrderItem> orderItems = new ArrayList<OrderItem>();
	...
	return manager.addOrder(email, orderItems);
}
\end{lstlisting}

Druhá metoda zůstane nezměněna. Aplikace se nyní bude chovat podle očekávání.


\section{JMS}
Běžná komunikace mezi komponentami probíhá synchronním způsobem. Jedna komponenta zavolá druhou a čeká než ta dokončí svůj běh před tím, než může pokračovat. Další podmínkou je, že jak volající, tak volaná komponenta musí být dostupná.

Pomocí JMS jsme schopni zajistit asynchronní komunikaci mezi heterogenními systémy. Producent zpráv nemá žádné informace o konzumentech ani o tom, jak s obdrženými zprávami nakládají. Konzument nemusí být v době odeslání zprávy k dispozici a může si zprávu vyzvednout později.

Ke konzumaci zpráv jsou určeny MDB. MDB jsou bezstavové, což znamená, že může na serveru současně běžet více instancí jedné beany, které konzumují zprávy z různých zdrojů. MDB čekají až dorazí zpráva, tu zkonzumují a zpracují. Mohou delegovat aplikační logiku na jiné bezstavové beany. MDB si nezachovávají žádný stav mezi přijetím jednotlivých zpráv.
JMS podporuje dva základní modely.

\vspace{5 mm}\noindent{\bf Bod-k-bodu}
\\\indent V tomto modelu mluvíme o producentovi a konzumentovi zpráv jako o odesílateli a příjemci. Zprávy odeslané odesilatelem jsou přechovávány ve frontě, dokud nejsou zkonzumovány nějakým příjemcem nebo dokud nevyprší jejich platnost. Za podmínky, že není žádný příjemce k dispozici, zůstává zpráva uchována ve frontě. Jakmile je příjemce vytvořen, může přijmout všechny čekající zprávy. Fronta může mít i více příjemců, ale každá zpráva může být zkonzumována pouze jednou.

\vspace{5 mm}\noindent{\bf Publikovat-odebírat}
\\\indent U tohoto modelu hovoříme o producentovi jako o publicistovi a o konzumentech (zpravidla je jich více) jako o odběratelích. Aby mohl odběratel konzumovat zprávy produkované publicistou, musí se přihlásit na téma, na které publicista zprávy zasílá. Téma zachovává zprávy, dokud nejsou distribuovány všem odběratelům. V případě, že se konzument přihlásí nově k tématu, nemá přístup k dříve odeslaným zprávám, a stane-li po nějakou dobu neaktivním, nedostane se ke zprávám, které byly producentem během jeho pasivity zaslány (toto chování se dá změnit).

Samotná zpráva se skládá z hlavičky, vlastností a těla. Hlavička obsahuje standardní informace pro identifikaci a směrování zprávy. Vlastnosti zprávy obsahují dvojice klíč-hodnota, které může aplikace nastavovat a číst. Tyto vlastnosti slouží spíše jako doplněk hlavičky, protože umožňují, aby se klient na základě vlastností rozhodl, zda zprávu příjme, nebo ne. Samotné tělo zprávy je nepovinné. Tělo zprávy se může skládat z mapy, textového řetězce, objektu a nebo proudu bytů.

V naší aplikaci bylo JMS použito na vytvoření abstrakce skladníka, který uzavírá objednávky a vytváří z nich faktury. Pokaždé, když je vytvořena nová objednávka, je poslána zpráva obsahující její identifikátor. Tato zpráva je zachycena MDB, která deleguje uzavření objednávky na ostatní bezstavové beany. Samotné posílání zprávy vypadá následovně:

\begin{lstlisting}
@Resource(mappedName = "java:/JmsXA")
private ConnectionFactory connectionFactory;
@Resource(mappedName = "java:/queue/test")

private void noticeStoreman(Long orderId) {
	Connection connection = null;
	connection = connectionFactory.createConnection();
		Session session = connection.createSession(false, Session.AUTO_ACKNOWLEDGE);
	MessageProducer messageProducer = session.createProducer(queue);
	connection.start();
	MapMessage message = session.createMapMessage();
	message.setStringProperty("type", "CLOSE_ORDER");
	message.setLongProperty("orderId", orderId);
	messageProducer.send(message);
	... 
\end{lstlisting}

část kódu příjemce potom vypadá takto:

\begin{lstlisting}
@MessageDriven(name = "StoremanMDB", activationConfig = {
@ActivationConfigProperty(propertyName = "destinationType", 
	propertyValue = "javax.jms.Queue"),
@ActivationConfigProperty(propertyName = "destination", 
	propertyValue = "queue/test"),
@ActivationConfigProperty(propertyName = "acknowledgeMode", 
	propertyValue = "Auto-acknowledge")})
public class StoremanMDB implements MessageListener {
	@Override
	public void onMessage(Message rcvMessage) {
	if (rcvMessage instanceof MapMessage) {
		MapMessage msg = (MapMessage) rcvMessage;
		StoremanMessage message =
		StoremanMessage.valueOf(msg.getStringProperty("type"));
	switch (message) {
			case CLOSE_ORDER:
			long id = msg.getLongProperty("orderId");
			// fakturuj
		...
}
\end{lstlisting}

\section{Časovaná služba}
Jedny z požadavků na aplikaci byly, aby aplikace byla schopná běžet dlouhodobě a jednotlivé komponenty bylo možné testovat izolovaně. 

K tomu, aby mohla aplikace běžet dlouhodobě, bylo zapotřebí zajistit postupné promazávání objednávek v databázi, aby nedošlo k nárůstu velikostí tabulek zaznamenávajících objednávky a faktury do enormních hodnot. K tomuto je použita třída Controller a její časovaná služba. Třída Controller je ve skutečnosti singletonem s anotací @Startup, které nám zaručí, že bude naše beana inicializována již během startování aplikace automaticky. Samotná časová služba je vyvolána anotací @Schedule(minute = "*", hour = "*"), která zaručí, že je metoda s touto anotací volána každou minutu běhu aplikace. 

Jakožto singleton je třída Controller zároveň ideálním kandidátem pro umístění všech konfiguračních prvků aplikace (ovládání skladníka, ovládání JMS, ovládání automatického čištění databáze a jiných).


\chapter{Prezentační vrstva}
K tomu, abychom umožnili uživateli pracovat s naší aplikací, je zapotřebí vytvořit nějaké uživatelské rozhraní. V naší aplikaci použijeme webové rozhraní, s kterým může uživatel komunikovat pomocí webového prohlížeče. Na tvorbu tohoto rozhraní využijeme specifikaci JSF (JavaServer Faces). 

JSF představuje nový model pro tvorbu dynamických stránek v Java EE vycházející z návrhového vzoru model-pohled-řadič (MVC). Dává programátorům možnost tvořit webové rozhraní pomocí komponent, událostí, managed bean a vzájemné interakce mezi nimi, čímž usnadňuje a zrychluje vývoj celého rozhraní~\cite{goncalves09}. Pokud bychom si nevystačili se základní sadou komponent, kterou nám jádro JSF poskytuje, máme k dispozici mnoho knihoven, kterými můžeme tuto sadu rozšířit (PrimeFaces, RichFaces a jiné). JSF 2.0 přineslo ve své nové verzi mnoho novinek jako:

\begin{itemize}
  \item využití faceletů namísto JSP jako hlavní technologie na tvorbu stránek
  \item nový způsob nakládání se zdroji
  \item rozšíření sady rozsahů, které mohou komponenty používat (nově přibyly: rozsah pro komponentu a rozsah pro daný pohled)
  \item zjednodušený vývoj validátorů, konvertorů a managed bean pomocí anotací
  \item přivlastnění principu konvence nad konfigurací, což znamená méně XML dokumentů (ty jsou nyní nepovinné)
  \item podpora pro AJAX
  \item jednoduchý vývoj nových komponent
\end{itemize}

\section{Facelety}
Facelety jsou alternativou a náhradou JSP. Poskytují nám možnost tvořit stránky v XML stylu, pomocí XHTML dokumentů, což sebou přináší výhodu v podobě možnosti validace stránek použitím standardních nástrojů pro XML, definování vlastních značek a možnost využít objektový model dokumentu (DOM). Další z mnoha výhod, které nám facelety poskytují, je schopnost vytvořit společný vzhled pro více stránek pomocí šablon. 

\section{Jazyk výrazů}
Chceme-li na našich stránkách vypsat hodnotu nějaké proměnné nebo přistoupit k atributu určitého objektu, můžeme takto učinit pomocí jazyka výrazů. Základní syntaxí tohoto jazyka je \#\{výraz\} (u JSP bylo zvykem používat také \$\{výraz\}, který vyhodnotil výraz okamžitě, toto se ovšem nehodí do životního cyklu JSF stránek). Pomocí tečkové notace můžeme budovat složitější výrazy. Jazyk výrazů nám poskytuje aritmetické, relační a logické operátory, díky kterým můžeme tvořit složitější výrazy.

\section{AJAX}
AJAX (dříve asynchronní JavaScript a XML) je soubor technik umožňujících nám vytvářet nenáročné, uživatelsky přívětivé a interaktivní webové aplikace. U tradičních webových aplikací musí prohlížeč požádat server o celý HTML dokument, což způsobuje překreslení celé stránky i při akcích, které toto chování nevyžadují. Za podmínky, že je tato stránka netriviálně veliká, dochází ke zbytečnému plýtvání šířkou pásma a zpomalení celé aplikace. AJAX používá asynchronní přenos dat mezi prohlížečem a serverem k získání pouze potřebné části informace namísto celé stránky, tato část stránky je poté pomocí JavaScriptu překreslena, což činí aplikaci plynulejší a rychlejší. Toto chování je nám umožněno i díky objektovému modelu dokumentu, pomocí kterého můžeme dynamicky pracovat s daty.

\begin{lstlisting}
<p:commandButton value="Remove" 
action="#{productBean.deleteProduct(_product.productName)}"
update="@form" />
\end{lstlisting}

Výše uvedený kus kódu říká JSF, že chceme po stisknutí tlačítka překreslit pouze formulář, ve kterém se tato komponenta vyskytuje. Pro přesnou specifikaci komponent, které mají být akcí ovlivněny, můžeme použít:
\begin{itemize}
  \item {\bf @all} - překreslí všechny komponenty na dané stránce 
  \item {\bf @none} - žádná komponenta nebude překreslena
  \item {\bf @this} - překreslí pouze tu komponentu, která vyvolala požadavek
  \item {\bf @form} - překreslí všechny komponenty z formuláře, ve kterém se komponenta, která vyvolala požadavek, nachází
  \item {\bf identifikátor komponenty} - překreslí jednu nebo více komponent specifikovaných pomocí jejich identifikátoru
  \item {\bf jazyk výrazů} - komponenty, které se budou vykreslovat, se mohou měnit za běhu podle toho, zda se budou jejich identifikátory vyskytovat v kolekci textových řetězců, ke které můžeme přistupovat pomocí jazyka výrazů.
\end{itemize}

\section{Managed Beany} 
Managed beany tvoří centrální bod webových aplikací. Managed beany mohou vykonávat aplikační logiku (nebo operace delegovat na EJB), obstarávat navigaci mezi stránkami, zajišťovat validaci a konvertování dat, ale také jejich přechovávání. K tomu, aby se ze třídy stala Managed beanou, je nutné, aby:
\begin{itemize}
  \item byla označena anotací @Named (díky této anotaci můžeme nyní k naší beaně přistupovat pomocí jazyka výrazů)
  \item měla definovaný rozsah~\footnote{jelikož mají managed beany většinou nastavený rozsah na požadavek, přináší nám CDI novou anotaci @Model, která v sobě zapouzdřuje anotace @Named a @RequestScoped}
  \item byla deklarována jako veřejná
\end{itemize}

\begin{figure}[!ht]
\centering
\includegraphics[width=0.35\textwidth]{frontendmodel}
\caption{Model prezentační vrstvy vytvořené za pomoci JSF}
\label{img:frontend}
\end{figure}

Kromě volání metod managed beany, jsme schopni pomocí jazyka výrazů pracovat také s jejími atributy. Ke zpřístupnění těchto atributů je zapotřebí, aby byly definovány metody get a set, které jsou implicitně volány. K atributům managed beany přistupujeme standardní tečkovou notací. Stejně jako každá jiná komponenta v Java EE 6 má i managed beana definovaný životní cyklus. K dispozici jsou nám anotace @PostConstruct a @PreDestroy (viz obrázek~\ref{img:mdb})

\begin{figure}[!ht]
\centering
\includegraphics[width=0.5\textwidth]{lifeManaged}
\caption{Model životního cyklu MDB}
\label{img:mdb}
\end{figure}

Metoda s anotací @PostConstruct je volána serverem okamžitě po vytvoření nové instance managed beany. Po zavolání této metody, je beana připravena obsluhovat požadavky uživatele. Předtím, než je beana odstraněna z paměti, je volána metoda @PreDestroy. Tyto metody mohou být použity například na vytvoření a uvolnění externích zdrojů.

\section{Zobrazování zpráv uživatelům}
\label{jsfMessage}
Má-li být naše aplikace robustní, měla by umět reagovat na nečekané události. Součástí reakce je poskytnutí adekvátních informací uživateli. Neočekávanými událostmi mohou být: chyba v aplikaci (selhání aplikační logiky, selhání sítě, selhání databáze) a nebo chyba uživatele při práci s aplikací (zadání nepovolených hodnot do formuláře). V prvním případě je vhodné přesměrovat uživatele na stránku s požadavkem o zopakování dotazu později. V druhém případě je vhodné uživatele upozornit na místo, ve kterém udělal chybu. K vyprodukování zprávy nám stačí použít v naší třídě následující metodu:

\begin{lstlisting}
private void addMessage(String summary) {
	FacesMessage message = new FacesMessage(FacesMessage.SEVERITY_WARN, 
		summary, null);
	FacesContext.getCurrentInstance().addMessage(null, message);
}
\end{lstlisting}

K zobrazení zprávy na naší stránce můžeme využít například komponentu z knihovny PrimeFaces:
\begin{lstlisting}
<p:growl id="messages" />
\end{lstlisting}

Jak bude vypadat odezva uživateli při pokusu o přihlášení s prázdným přihlašovacím formulářem, můžeme vidět na obrázku~\ref{img:messages}. Vypsané zprávy jsou ty samé, které jsme definovali u našich entit jako omezení na datovém modelu.  

\begin{figure}[!ht]
\centering
\includegraphics[width=0.7\textwidth]{messages}
\caption{Zobrazení zpráv uživateli}
\label{img:messages}
\end{figure}

\chapter{Webové služby a autentizace}
\section{REST}
\label{rest}
Další alternativou k interakci s aplikací je použití rozhraní REST. REST je koncept pro design distribuované architektury, kdy části programu běží na různých strojích a ke své komunikaci využívají síť. REST je, na rozdíl od známějších XML-RPC či SOAP, orientován datově, nikoli procedurálně. Webové služby definují vzdálené procedury a protokol pro jejich volání, REST určuje, jak se přistupuje k datům~\cite{maly09}. Základními principy RESTu jsou:

\begin{itemize}
  \item stav aplikace a chování je vyjádřen zdrojem
  \item každý zdroj má unikátní identifikátor (URL, URN)
  \item je definován jednotný přístup pro získání a manipulaci se zdroji
  \item zdroj může mít různé reprezentace (XML, HTML, JSON, SVG, PDF)
\end{itemize}

REST definuje čtyři základní metody a to: vytvoření dat, získání dat, změnu dat a smazání dat. Tyto metody jsou implementovány pomocí odpovídajících metod HTTP protokolu. Námi poskytované REST rozhraní by mělo splňovat několik zásad:

\begin{itemize}
  \item {\bf jednotnost rozhraní} - rozhraní by mělo být jednotvárné napříč celou aplikací
  \item {\bf adresovatelnost} - každá informace, která má nějakou hodnotu, by měla být zpřístupněna jako zdroj a měla by mít vlastní URI
  \item {\bf souvislost} - existuje-li nějaký vztah mezi dvěma různými zdroji, měly by na sebe tyto zdroje vzájemně odkazovat
  \item {\bf bezstavovost} - požadavky klientů by se neměly navzájem ovlivňovat 
\end{itemize}

Ke zpřístupnění našeho rozhraní nejprve potřebujeme určit, na které adrese naší aplikace budou jednotlivé zdroje přístupné. K tomu slouží třída JaxRsActivator, která toto definuje pomocí anotace @ApplicationPath.

\begin{lstlisting}
@ApplicationPath("/rest")
public class JaxRsActivator extends Application {
}
\end{lstlisting}

Následně definujeme pro každý typ zdroje vlastní třídu, která poskytne veškeré služby vztahující se k tomuto zdroji. Pojďme si ukázat, jak probíhá vytvoření a následné získání zákazníka v naší aplikaci pomocí RESTového rozhraní.

Nejprve definujeme třídu, ve které zpřístupníme služby týkající se zákazníků.

\begin{lstlisting}
@Path("/products")
@RequestScoped
public class ProductResourceRESTService {
...
}
\end{lstlisting}

Pomocí anotace @Path deklarujeme adresu, na které budeme moci přistupovat k tomuto zdroji. Definujeme metodu pro vkládání nových záznamů. Tato metoda musí mít anotaci @POST. Uvnitř metody požadavek klienta zpracujeme, validujeme data, vytvoříme nového zákazníka a vrátíme patřičnou odezvu.

\begin{lstlisting}
@POST
@Path("/create/{name:[A-Za-z0-9]*}")
public Response createProduct(@PathParam("name") String name,
	@QueryParam("price") @DefaultValue("1") Long price,
	@QueryParam("type") @DefaultValue("1") Integer type,
	@QueryParam("stored") @DefaultValue("100") Long stored) {
	...
}
\end{lstlisting}

K vytvoření klienta našeho webového rozhraní můžeme použít knihoven jako Jersey nebo RESTeasy, my si pro naši malou ukázku vystačíme s cURLem. CURL, což je konzolový nástroj pro přenos dat pomocí protokolů jako HTTP, FTP a jiných. Pomocí příkazu:

\begin{lstlisting}
curl -X POST 'http://localhost:8080/web/
	rest/products/create/lednice?price=200&stored=100&type=4'
\end{lstlisting}

vytvoříme nový produkt. Pro získání produktu zpět nejprve musíme definovat metodu s anotací @GET:

\begin{lstlisting}
@GET
@Path("/name/{byName:[A-Za-z0-9]*}")
@Produces(MediaType.APPLICATION_XML)
public ProductDao lookupProductByName(@PathParam("byName") String name) {
	return new ProductDao(productManager.getProductByName(name));
}
\end{lstlisting}

Tato metoda vrací instanci třídy ProductDao, která je součástí vrstvy objektů určených pro práci s RESTovým rozhraním, která nám zaručí bezstavovost jednotlivých požadavků. 

Třídy vrstvy DAO jsou označeny anotaci @XMLRootElement, což je anotace architektury JAXB, která umožňuje mapovat Java třídy na XML dokumenty. Díky JAXB můžeme nastavit XML jako návratový typ dat, který tato metoda poskytuje. Třídy vrstvy DAO mají pouze jednosměrné asociace. Byla přidána nová třída CustomerLiteDao, která neobsahuje referenci ani na objednávky ani na faktury, díky tomu jsme schopni vypsat zákazníka i se seznamem jeho objednávek, aniž bychom se zacyklili.

Pro získání produktu můžeme jako klienta použít obyčejný webový prohlížeč, který volá metodu GET implicitně. Náš nově vytvořený produkt je k dispozici na adrese http://localhost:8080/web/rest/products/name/lednice.
Po zadání této adresy do prohlížeče nám bude serverem vrácena XML reprezentace daného produktu.

\begin{lstlisting}
<product>
	category>TYPE4</category>
	id>1</id><name>lednice</name>
	price>200</price>
	reserved>0</reserved>
	stored>100</stored>
</product>
\end{lstlisting}

RESTové rozhraní naší aplikace poskytuje téměř shodnou funkcionalitu s webovým. Dovoluje nám práci s objekty, generování náhodných objektů i konfiguraci aplikace samotné.

\section{Autentizace}
Jelikož se jedná o testovací aplikaci, byl zvolen zjednodušený mechanismus autentizace. Jestliže by byla náhodně vygenerovaná hesla ukládána ve formě hashe, nemohli bychom s aplikací pracovat v reálném čase. Hesla jsou proto zobrazena ve formě holého textu. Nyní si popíšeme celý koncept přihlašování se do aplikace (viz obrázek~\ref{img:identity}).

\begin{figure}[!ht]
\centering
\includegraphics[width=0.65\textwidth]{loggingmedol}
\caption{Model autentizace zákazníka a práce s jeho identitou}
\label{img:identity}
\end{figure}

Základ tvoří třída Identity, což je managed beana s rozsahem nastaveným na dané sezení. Poté, co uživatel zadá přihlašovací údaje do formuláře (ty musí být validní, jinak k přihlášení nedojde), který se vyskytuje v naší šabloně, a klikne na tlačítko přihlásit se, je zavolána akce třídy LoginBean. Tato managed beana, s rozsahem na požadavek, předá přihlašovací údaje třídě Identity, která si tyto údaje zapamatuje a vytvoří novou instanci třídy BasketBean (v rámci jednoho sezení se můžeme přihlásit za více uživatelů, proto je důležité aby měl každý z nich po přihlášení prázdný košík). Třída BasketBean má stejně jako třída Identity nastavený rozsah na dané sezení. Zákazník přidává nové produkty do košíku pomocí třídy BasketBeany a jakmile se rozhodne objednávku uzavřít, je z této třídy zavolána patřičná metoda bezstavové beany OrderManager, kterou jsme si do třídy BasketBean injektovali. Jako parametry metody na přidání nové objednávky jsou předány: kolekce objednaných produktů a email zákazníka (který je uložen ve třídě Identity)~\footnote{záměrně jsou vynechány detaily jako validace dat a autorizace, jelikož nejsou pro tuto ukázku podstatné}.

\chapter{P}
\section{Nástroje}
\subsection{Apache JMeter}
umoznuje nam odstrelit ve skriptovacim modulu obrazky
\subsection{Monitorování Honze merak pameti a modul Jmeteru}
GC rozdeleni pameti, co tam vlastne merim, pustim to vzdycky pred testem a okamzite po testu
\subsection{JBoss AS 7.1.1, 7.2 a glass fish}

\section{Testování}

\subsection{JVM}
1.6 / 1.7
\subsection{Velikost haldy a treba nejake pooly}
\subsection{Rozdil mezi JMS a direct u JBossu}
\subsection{Rozdil mezi glassfishem a jbossem direct}


\chapter*{Závěr}

Nedostatek výsledků ke srovnání komerčních serverů, žádná možnost srovnání 

Problém s transakcema, a spatnou architekturou (smazano)

testovat na vice stojich (as na jednom, db na druhem..)

K tomu aby bylo dodrzen pristu, ze vykonostni testy nejsou testy funcionality lblblo byla poslouzila sada jendotkovych testu ktera poskytla urcitou miru jistoty ze funkcionalita je ok i . Kolikrat me az samotneho zarazilo, jak dokaze mala zmena na jednom miste ovlivnit uplne jinou cast aplikace nastesti diky testum se tyto bla podarili vcas najit aj kdyz najit chybu kolikrat bylo dost frustrujici dokazu si predstavit, ze bez testu by bylo jeste frustrujicim.

Muzeme si zaplatit a pouzit SPEC nebo Day trader. Day trader sice pouziva veci jako entity beany a JSP, ale jelikoz vychazi jako ukazkova aplikace apache Geronimo ktery je (???) certifikovany EE, bude day trader updatovat aby vzdy podporoval nejnovejsi specifikace obsazene ve vzdy posledni verzy Java EE platf.

Zanorovani JSF komponent

Testovani: Rozdily mezi servery, neni to jen tak vzit archiv a nahrat ho na jiny server, je potreba sjednotit i veci jako hibernate a eclipse link
uprava na transient v Identity bean co to je proc to je
hashe, nevim kdy a proc ale jelikos sem pouzival jen seznamy tak se volalo asi jen equals protoze sem mel celou dobu cykly v hash code a na JBossu to nevadilo ani jednou

ruzne chovani, eclips linkem me to nejelo. // ruzni provideri specifikaci, nejak to prelozit.

do budoucna:
mozna nejake prepinace na druh zateze treba mezi rezimem kdy budu jenom zapisovat a rezimem kdy budu jen cist
vytvorit kompletni sadu testovacich skriptu pomoci Jmeteru aby clovek nemusel nic delat i s merenim, melo by to jit pouzet z cmd : Jelikoz je aplikace plne interaktivni, nemelo by byt tvoreni skriptu vetsim problemem (napr pomoci skriptovaciho modulu JMeteru) a nasledne to upravit a treba pomoci nejakych souboru nebo promennych s tim neco uz delat. Jmeter urcite nebude nejkomplexnejsim testvacim nastrojem, vybral sem si ho hlavne pro jeho jednoduchost tudiz mozna to sou daleko lepsi nastroje este.

Pokud by to nepomohlo, mohlo by to vyresit napriklad odstraneni obousmerne vazby mezi Order a Invoice 
Pokud uzaviram objednavku je to moc dlouha transakcni davka. Mozna by to slo upravim nejakym nastavenim serveru, ale veskere me pokusy selhaly

\appendix

\chapter{Nasazení aplikace} 
požadavky:
\begin{itemize}
  \item Git
  \item Apache Maven (verze alespoň 3.0)
  \item JDK (verze alespoň 1.6.0)
\end{itemize}
\section{Příprava aplikačního serveru}
Ke spuštění aplikace nejprve potřebujeme aplikační server, na který budeme moci naši aplikaci nasadit. Tento server musí být certifikován na Java EE plný profil, aby bylo možné využívat služeb JMS. Aplikace byla vyvíjena za použití aplikačního serveru JBoss AS 7.2.0 Alpha. Jedná se o server s otevřenými zdrojovými kódy, který je k dispozici volně ke staženi~\footnote{v tomto návodu budeme dále předpokládat, že je použit server JBoss AS verze alespoň 7.1.1}. Server lze stáhnout buď na adrese \href{http://www.jboss.org/jbossas/downloads/}{http://www.jboss.org/jbossas/downloads/} nebo si sestavit vlastní ze zdrojových kódů:
\begin{itemize}
  \item stáhneme si zdrojové kódy z repozitáře:
\begin{lstlisting}
$ git clone git://github.com/jbossas/jboss-as.git
\end{lstlisting}
  \item a server sestavíme:
\begin{lstlisting}
$ cd jboss-as/
$ mvn install
\end{lstlisting}
\end{itemize}
Naši novou instanci serveru můžeme nyní najít v adresáři: 
\begin{lstlisting}
jboss-as/build/target/jboss-as-7.2.0.Alpha1-SNAPSHOT
\end{lstlisting}
Pro zjednodušení práce si vytvořme novou proměnnou prostředí \$JBOSS\_HOME odkazující na tento adresář.

\subsection{Datový zdroj}
K tomu, aby mohla naše aplikace pracovat s databází, musíme serveru nastavit datový zdroj. V této ukázce použijeme databázi MySQL.
Nejprve musíme přidat nový modul~\footnote{JBoss AS 7 poskytuje systém modulů, aby zajistil isolaci při načítání tříd do paměti~\cite{datasource}}
\begin{lstlisting}
$ cd $JBOSS_HOME
$ mkdir -p modules/com/mysql/main
$ cd modules/com/mysql/main
\end{lstlisting}
Knihovnu s ovladačem k databázi zkopírujeme na tuto lokaci a vytvoříme nový soubor module.xml, s následujícím obsahem~\footnote{verzi knihovny upravíme tak, aby seděla s tou, kterou se chystáme použít}:
\begin{lstlisting}
<?xml version="1.0" encoding="UTF-8"?>
	<resources>
	<resource-root path="mysql-connector-java-5.1.22-bin.jar"/>
	</resources>
	<dependencies>
		<module name="javax.api"/>
	</dependencies>
</module>
\end{lstlisting}
Nyní, když máme vytvořený modul, musíme na něj udělat odkaz v hlavním konfiguračním souboru serveru:
\begin{lstlisting}
	$ cd $JBOSS_HOME/standalone/configuration
	$ vim standalone-full.xml
\end{lstlisting}
Do sekce s ovladači přidáme:
\begin{lstlisting}
<drivers>
	<driver name="mysql" module="com.mysql"/>
	<driver name="h2" module="com.h2database.h2">
        <xa-datasource-class>
		org.h2.jdbcx.JdbcDataSource
        </xa-datasource-class>
	</driver>
</drivers>
\end{lstlisting}
Ve stejném souboru musíme nastavit také datový zdroj:
\begin{lstlisting}
<datasource
        jndi-name="java:/mydb" pool-name="my_pool"
        enabled="true" jta="true"
        use-java-context="true" use-ccm="true">
    <connection-url>
        jdbc:mysql://localhost:3306/mydb
    </connection-url>
    <driver>
        mysql
    </driver>
    <security>
        <user-name>
            root
        </user-name>
        <password>
        </password>
    </security>
    <statement>
        <prepared-statement-cache-size>
            100
        </prepared-statement-cache-size>
        <share-prepared-statements/>
    </statement>
</datasource>
\end{lstlisting}
\subsection{JMS}
K využití služeb naší MDB musíme nastavit frontu, která bude uchovávat zprávy vyprodukované naší aplikací. Ke konfiguraci JMS fronty využijeme CLI (Command Line Interface), což je jedno z rozhraní na spravování serveru, které JBoss AS 7 nabízí. Nejprve spustíme server:
\begin{lstlisting}
$ cd $JBOSS_HOME/bin
./standalone.sh -c standalone-full.xml
\end{lstlisting}
Jakmile server nastartuje, můžeme se přihlásit k CLI a frontu zaregistrovat:
\begin{lstlisting}
./jboss-cli.sh -c
[standalone@localhost:9999 /] jms-queue add --queue-address=testQueue\
> --entries=queue/test,java:jboss/exported/jms/queue/test
\end{lstlisting}

\section{Spuštění aplikace}
Nyní, když máme spuštěný server, nám už nic nebrání v nasazení aplikace. Nejprve si stáhneme zdrojové kódy aplikace z repozitáře, který lze najít na adrese \href{https://github.com/pkremens/muni-eshop}{https://github.com/pkremens/muni-eshop}:
\begin{lstlisting}
$ git clone git://github.com/pkremens/muni-eshop.git
$ cd muni-eshop/Eshop
\end{lstlisting}

\subsection{Testovací sada aplikace}
Aplikace obsahuje sadu jednoduchých testů, které ověřují základní funkcionalitu a zároveň nám zaručí, že je naše běhové prostředí připraveno na její nasazení. 
Testovací sadu spustíme pomocí příkazu:
\begin{lstlisting}
$ mvn test
\end{lstlisting}
V případě, že testy proběhly úspěšně, měli bychom na konzoli vidět následující výpis~\footnote{aplikaci se můžeme pokusit nasadit i za podmínky, že by některý z testů selhal díky argumentu -DskipTest=true, ale je doporučeno pokusit se zjistit příčinu chyb a tuto příčinu odstranit}:
\begin{lstlisting}
-------------------------------------------------------
 T E S T S
-------------------------------------------------------
...
Results :

Tests run: 35, Failures: 0, Errors: 0, Skipped: 0

[INFO] ------------------------------------------------------------------------
[INFO] BUILD SUCCESS
[INFO] ------------------------------------------------------------------------
[INFO] Total time: 2:22.193s
[INFO] Finished at: Fri Jan 04 03:21:53 CET 2013
[INFO] Final Memory: 13M/207M
[INFO] ------------------------------------------------------------------------
\end{lstlisting}

\subsection{Nasazení aplikace}
K nasazení aplikace použijeme nástroj Maven a jeho pluginy, které za nás vykonají veškerou:
\begin{lstlisting}
$ mvn install jboss-as:deploy -DskipTests=true
...
[INFO] ------------------------------------------------------------------------
[INFO] Reactor Summary:
[INFO] 
[INFO] Eshop: ............................................ SUCCESS [0.726s]
[INFO] Eshop: EJB Module ................................. SUCCESS [4.202s]
[INFO] Eshop: Web Module ................................. SUCCESS [2.344s]
[INFO] Eshop: Application Module ......................... SUCCESS [12.369s]
[INFO] ------------------------------------------------------------------------
[INFO] BUILD SUCCESS
[INFO] ------------------------------------------------------------------------
[INFO] Total time: 20.168s
[INFO] Finished at: Thu Jan 03 20:17:36 CET 2013
[INFO] Final Memory: 42M/462M
[INFO] ------------------------------------------------------------------------
\end{lstlisting}
S webovým rozhraním aplikace můžeme pracovat na adrese:
\begin{lstlisting}
http://localhost:8080/web/index.html
\end{lstlisting}
Máme-li k dispozici nějaký nástroj na generování zátěže, můžeme ke generování použít servlet na adrese:
\begin{lstlisting}
http://localhost:8080/web/ShoppingServlet
\end{lstlisting}

\subsection{OpenShift}
Odlehčená verze aplikace~\footnote{aplikace byla zabalena jako webový archiv, není použito JMS a aplikace používa odlehčenou H2 databázi} byla nasazena na platformě OpenShift společnosti Red Hat a je možné s ní pracovat na adrese:
\begin{lstlisting}
http://opensh-pkremens.rhcloud.com/
\end{lstlisting}

\chapter{Seznam funkcí dostupných z webového rozhraní}
Následující výčet zahrnuje seznam funkcí, které jsou dostupné z webového rozhraní aplikace.\\

\noindent{\bf index.xhtml}:
\begin{itemize}  \itemsep0em
  \item smazat obsah tabulek
  \item smazat tabulky objednávek a faktur
  \item zobrazit přehled o naplnění tabulek
  \item zapnout nebo vypnout službu automatického mazání objednávek a faktur
  \item zapnout nebo vypnout službu automatické fakturace objednávek
  \item zapnout nebo vypnout službu automatické fakturace objednávek pomocí JMS
  \item vygenerovat $n$ náhodných zákazníků 
  \item smazat tabulku zákazníků
  \item vygenerovat $n$ náhodných produktů
  \item smazat tabulku produktů
\end{itemize}
{\bf template.xhtml}:
\begin{itemize}  \itemsep0em
  \item přihlásit se 
  \item odhlásit se
\end{itemize}
{\bf customers.xhtml}:
\begin{itemize}  \itemsep0em
  \item vytvořit zákazníka
  \item vytvořit a přihlásit zákazníka
  \item vygenerovat náhodného zákazníka
  \item procházet seznam zákazníků
  \item smazat zákazníka
  \item smazat všechny zákazníky
  \item získat XML reprezentaci zákazníka (REST)
  \item získat XML reprezentaci všech zákazníků (REST)
\end{itemize}
{\bf products.xhtml}:
\begin{itemize}  \itemsep0em
  \item vytvořit produkt
  \item vygenerovat náhodný produkt
  \item procházet seznam produktů
  \item smazat produkt
  \item smazat všechny produkty
  \item získat XML reprezentaci produktu (REST)
  \item získat XML reprezentaci všech produktů (REST)
  \item přidat produkt do košíku
\end{itemize}
{\bf basket.xhtml}:
\begin{itemize}  \itemsep0em
  \item odebrat produkt z košíku
  \item přidat a odebrat jeden kus konkrétního produktu
  \item přidat a odebrat pět kusů konkrétního produktu
  \item vyprázdnit košík
  \item dokončit objednávku
\end{itemize}
{\bf orders.xhtml}:
\begin{itemize}  \itemsep0em
  \item vygenerovat náhodnou objednávku
  \item fakturovat objednávku
  \item procházet seznam objednávek
  \item smazat všechny objednávky
  \item získat XML reprezentaci objednávky (REST)
  \item získat XML reprezentaci všech objednávek (REST)
\end{itemize}
{\bf invoice.xhtml}:
\begin{itemize}  \itemsep0em
  \item procházet seznamem faktur
  \item získat XML reprezentaci faktury (REST)
  \item získat XML reprezentaci všech faktur (REST)
  \item smazat všechny faktury
\end{itemize}
{\bf history.xhtml}:
\begin{itemize}  \itemsep0em
  \item procházet historii objednávek přihlášeného zákazníka 
  \item zobrazit detail objednávky (viz obrázek~\ref{img:orderDetail})
  \item skrýt detail objednávky
\end{itemize}

\begin{figure}[!ht]
\centering
\includegraphics[width=0.7\textwidth]{orderDetail}
\caption{Zobrazeni detailu objednávky}
\label{img:orderDetail}
\end{figure}


%\ttfamily
%\texttt{adsadsad} \\
%\normalfont



\begin{thebibliography}{99}
%The \bibitem is to start a new reference.  Ensure that the cite_key is
%unique.  You don't need to put each element on a new line, but I did
%simply for readability.
\addcontentsline{toc}{section}{Literatura}

\bibitem{molyneaux09}
MOLYNEAUX, Ian.
\emph{: The Art of Application Performance Testing}.
1st ed. Sebasctopol, CA: O'Reilly Media, c2009, xii, 144 p. Theory in Practice (Sebascopol, Calif.). 
ISBN 987-059-6520-663.

\bibitem{haines06}
HAINES, Steven.
\emph{: Pro Java EE 5 Performance Management and Optimization}.
Berkley, Calif: Apress, 2006.
ISBN 978-159-0596-104.

\bibitem{barber11}
BARBER, Scott a MASON, Colin.
\emph{: Web Load Testing For Dummies}.
Compuware Special Edition. 111 River Street Hoboken, NJ 07030-5774: John Wiley \& Sons, Inc., 2011.
ISBN 978-1-118-16016-9.
	  
\bibitem{kodali06}
KODALI, Raghu R, WETHERBEE, Jonathan a ZADROZNY Peter.
\emph{: Beginning EJB 3 application development: from novice to professional}.
Berkeley, CA: Apress, c2006, xxviii, 481 p.
ISBN 978-159-0596-715. 

\bibitem{goncalves09}
GONCALVES, Antonio.
\emph{: Beginning Java EE 6 Platform with GlassFish 3: from novice to professional}.
New ed. Berkeley, CA: Apress, 2009.
ISBN 978-143-0219-545. 

\bibitem{maly09}
MALÝ, Martin.
\emph{: REST: architektura pro webové API}.
2009-08-03.
\textless\href{http://www.zdrojak.cz/clanky/rest-architektura-pro-webove-api/}{http://www.zdrojak.cz/clanky/rest-architektura-pro-webove-api}\textgreater.

\bibitem{bhatt09}
BHATT, NIRAJ.
\emph{: Performance Testing – Response vs. Latency vs. Throughput vs. Load vs. Scalability vs. Stress vs. Robustness}.
2009-09-17
\textless\href{http://nirajrules.wordpress.com/2009/09/17/measuring-performance-response-vs-latency-vs-throughput-vs-load-vs-scalability-vs-stress-vs-robustness/}{http://nirajrules.wordpress.com/2009/09/17/measuring-performance-response-vs-latency-vs-throughput-vs-load-vs-scalability-vs-stress-vs-robustness/}\textgreater.

\bibitem{moorthy09}
MOORTHY, Thiru
\emph{:  End User Response Time Vs Tool Response Time}.
2009-01-15.
\textless\href{http://perfhints.blogspot.cz/2009/01/end-user-response-time-vs-tool-response.html}{http://perfhints.blogspot.cz/2009/01/end-user-response-time-vs-tool-response.html}\textgreater.

\bibitem{sintes02}
SINTES, Tony.
\emph{: App server, Web server: What's the difference?}.
2002-08-23.
\textless\href{http://www.javaworld.com/javaqa/2002-08/01-qa-0823-appvswebserver.html?page=1}{http://www.javaworld.com/javaqa/2002-08/01-qa-0823-appvswebserver.html?page=1}\textgreater.

\bibitem{wikiServerComp}
\emph{Server (computing)}.
San Francisco (CA): Wikimedia Foundation, 2012-12-29.
\textless\href{http://en.wikipedia.org/wiki/Server\_(computing)}{http://en.wikipedia.org/wiki/Server\_(computing)}\textgreater.

\bibitem{daytrader}
\emph{Daytrader: A More Complex Application}.
Apache Geronimo, 2012-11-22.
\textless\href{https://cwiki.apache.org/GMOxDOC30/daytrader-a-more-complex-application.html}{https://cwiki.apache.org/GMOxDOC30/daytrader-a-more-complex-application.html}\textgreater.

\bibitem{weld}
KING, Gavin a MUIR, Pete.
\emph{Weld - JSR-299 Reference Implementation}.
2009-10-10.
\textless\href{http://docs.jboss.org/weld/reference/latest/en-US/html/}{http://docs.jboss.org/weld/reference/latest/en-US/html/}\textgreater.

\bibitem{wikiEJB}
\emph{Enterprise Java Beans}.
San Francisco (CA): Wikimedia Foundation, 2012-10-04.
\textless\href{http://cs.wikipedia.org/wiki/Enterprise\_Java\_Beans}{http://cs.wikipedia.org/wiki/Enterprise\_Java\_Beans}\textgreater.

\bibitem{wikiJSR}
\emph{Java Community Process}.
San Francisco (CA): Wikimedia Foundation, 2011-07-28.
\textless\href{http://cs.wikipedia.org/wiki/Java\_Community\_Process}{http://cs.wikipedia.org/wiki/Java\_Community\_Process}\textgreater.

\bibitem{wikiJavaEE}
\emph{Java EE}.
San Francisco (CA): Wikimedia Foundation, 2012-10-10.
\textless\href{http://cs.wikipedia.org/wiki/Java\_EE}{http://cs.wikipedia.org/wiki/Java\_EE}\textgreater.

\bibitem{loadTest}
\emph{Types of Tests}
RPM Solutions Pty Ltd, 2004-08-04.
\textless\href{http://www.loadtest.com.au/types\_of\_tests.htm}{http://www.loadtest.com.au/types\_of\_tests.htm}\textgreater.

\bibitem{wikiBench}
\emph{Benchmark (computing)}.
San Francisco (CA): Wikimedia Foundation, 2012-09-30.
\textless\href{http://en.wikipedia.org/wiki/Benchmark\_(computing)}{http://en.wikipedia.org/wiki/Benchmark\_(computing)}\textgreater.

\bibitem{sybase}
PONNACHATH, Prashanth.
\emph{Application Server Benchmarking Fundamentals}.
Sybase Inc.
\textless\href{http://www.sybase.com/content/1018914/Benchmarking\_fundamentals\_v3.pdf}{http://www.sybase.com/content/1018914/Benchmarking\_fundamentals\_v3.pdf}\textgreater.

\bibitem{performanceTuning}
\emph{JBoss AS 7 Performance tuning}.
2013-01-02
\textless\href{http://www.mastertheboss.com/jboss-performance/jboss-as-7-performance-tuning}{http://www.mastertheboss.com/jboss-performance/jboss-as-7-performance-tuning}\textgreater.

\bibitem{datasource}
\emph{Adding a MySQL datasource to JBoss AS 7}
2011-07-12
\textless\href{https://zorq.net/b/2011/07/12/adding-a-mysql-datasource-to-jboss-as-7/}{https://zorq.net/b/2011/07/12/adding-a-mysql-datasource-to-jboss-as-7/}\textgreater.

\end{thebibliography} %Must end the environment


% Následují další kapitoly a podkapitoly, popřípadě závěr, dodatky, 
% seznam literatury či použitých obrázků nebo tabulek.

% \bibliographystyle{plain}  % bibliografický styl 
% \bibliography{mujbisoubor} % soubor s citovanými
                           % položkami bibliografie 

\end{document}
